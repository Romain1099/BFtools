
\documentclass[12pt]{article}
\usepackage{geometry}

\geometry{paperwidth=18cm, paperheight=8.7cm, margin=0.2cm}

\usepackage{bfcours}
\usepackage{bfcours-fonts}
\usepackage{pagecolor} % Pour colorer le fond de la page
%\usetikzlibrary{backgrounds}
\usepackage{expl3} % Nécessaire pour utiliser le langage LaTeX3

% Retourne la valeur absolue d'un entier
\newcommand{\absnum}[1]{%
    \ifnum #1<0 \number\numexpr -#1\relax\else \number\numexpr #1\relax\fi%
}

% Retourne le minimum de deux entiers
\newcommand{\minnum}[2]{%
    \ifnum #1<#2 #1\else #2\fi%
}

% Calcul récursif du PGCD de deux entiers (les arguments doivent être entiers)
\newcommand{\gcdcalc}[2]{%
    \ifnum #2=0 %
    \absnum{#1}%
    \else %
    \gcdcalc{#2}{\the\numexpr #1 - (#1/#2)*#2 \relax}%
    \fi%
}
\ExplSyntaxOn
% Définition de la commande \mprefix
\cs_new:Npn \mprefix #1
{
    \int_case:nnF {#1}
    {
        {-12} { \text{pico} }
        {-9}  { \text{nano} }
        {-6}  { \text{micro} }
        {-3}  { \text{milli} }
        {-2}  { \text{centi} }
        {-1}  { \text{deci} }
        {0}   { \text{} } % Aucun préfixe pour la puissance 0
        {1}   { \text{deca} }
        {2}   { \text{hecto} }
        {3}   { \text{kilo} }
        {6}   { \text{mega} }
        {9}   { \text{giga} }
        {12}  { \text{tera} }
    }
    { \text{Valeur inconnue} } % Gestion des cas non pris en charge
}
\ExplSyntaxOff


% Calcul et affichage des diviseurs avec Lua
\begin{luacode*}
    function generate_divisors_table(n)
    local seen = {} -- Table pour suivre les nombres déjà rencontrés
    tex.print("\\noindent\\begin{tabular}{c|c}")
    for i = 1, n do
    if n % i == 0 then
    local complement = math.floor(n / i) -- Calcul de la partie entière
    -- Vérifier si le complément est déjà dans la table 'seen'
    if seen[complement] then break end
    -- Ajouter le diviseur et le complément dans la table
    seen[i] = true
    tex.print(i .. " & " .. complement .. " \\\\")
    end
    end
    tex.print("\\end{tabular}")
    end
\end{luacode*}

\newcommand{\divisorstable}[1]{%
    \directlua{generate_divisors_table(#1)}%
}

% Commande pour s'adapter en fonction du choix
\newcommand{\marchandcmd}[1]{%
    \IfStrEqCase{#1}{%
        {marchand}{\xdef\munite{articles}\def\intro{}\def\vend{vend}\def\tousSes{$\proportion$ de ses }\def\oeufs{$\quantite$\ articles}\def\labelvente{le nombre d'articles vendus}}
        {agriculteur}{\xdef\munite{tonnes}\def\intro{Suite à des intempéries,}\def\vend{perd}\def\tousSes{les $\proportion$èmes}\def\oeufs{sa récolte, estimée à \quantite\  tonnes}\def\labelvente{la masse perdue}}
        {boulanger}{\xdef\munite{baguettes}\def\intro{}\def\vend{vend}\def\tousSes{les $\proportion$èmes}\def\oeufs{de ses $\quantite$ baguettes}\def\labelvente{le nombre de baguettes vendues}}
    }[\textcolor{red}{\textbf{Métier inconnu : #1}}]% Si aucune correspondance
}

% Définition de la commande criteredivisibilite
\newcommand{\criteredivisibilite}[2]{%
    % #1 est le diviseur (n), #2 est le nombre à tester (N)
    \def\diviseur{#1}%
    \def\nombre{#2}%
    
    % Calcul du dernier chiffre du nombre
    \def\dernierchiffre{\intcalcMod{#2}{10}}
    
    % Calcul des deux derniers chiffres du nombre
    \def\deuxdernierchiffres{\intcalcMod{#2}{100}}
    
    % Vérifier quel diviseur a été spécifié
    \ifthenelse{\equal{\diviseur}{2}}{%
        % Cas 2
        \textbf{Critère de divisibilité par 2 :} Un nombre est divisible par 2 si son dernier chiffre est pair (0, 2, 4, 6 ou 8).
        
        \textbf{Application à \nombre :} Le dernier chiffre de \nombre{} est \dernierchiffre.
        
        \ifthenelse{\equal{\intcalcMod{\nombre}{2}}{0}}{%
            Comme le dernier chiffre est \dernierchiffre{} (qui est pair), \nombre{} est divisible par 2.
            }{%
            Comme le dernier chiffre est \dernierchiffre{} (qui n'est pas pair), \nombre{} n'est pas divisible par 2.
        }%
        }{%
        \ifthenelse{\equal{\diviseur}{3}}{%
            % Cas 3
            \textbf{Critère de divisibilité par 3 :} Un nombre est divisible par 3 si la somme de ses chiffres est divisible par 3.
            
            \textbf{Application à \nombre :}
            
            Pour vérifier si \nombre{} est divisible par 3, calculons la somme de ses chiffres.
            
            % Dans un cas réel, on calculerait la somme des chiffres
            \ifthenelse{\equal{\intcalcMod{\nombre}{3}}{0}}{%
                La somme des chiffres de \nombre{} est divisible par 3, donc \nombre{} est divisible par 3.
                }{%
                La somme des chiffres de \nombre{} n'est pas divisible par 3, donc \nombre{} n'est pas divisible par 3.
            }%
            }{%
            \ifthenelse{\equal{\diviseur}{4}}{%
                % Cas 4
                \textbf{Critère de divisibilité par 4 :} Un nombre est divisible par 4 si le nombre formé par ses deux derniers chiffres est divisible par 4.
                
                \textbf{Application à \nombre :}
                \ifthenelse{\nombre < 100}{%
                    Comme \nombre{} est inférieur à 100, nous vérifions directement si \nombre{} est divisible par 4.
                    
                    \ifthenelse{\equal{\intcalcMod{\nombre}{4}}{0}}{%
                        Le reste de la division de \nombre{} par 4 est 0, donc \nombre{} est divisible par 4.
                        }{%
                        Le reste de la division de \nombre{} par 4 est \intcalcMod{\nombre}{4}, donc \nombre{} n'est pas divisible par 4.
                    }%
                    }{%
                    Les deux derniers chiffres de \nombre{} forment le nombre \deuxdernierchiffres.
                    
                    \ifthenelse{\equal{\intcalcMod{\deuxdernierchiffres}{4}}{0}}{%
                        Comme \deuxdernierchiffres{} est divisible par 4, \nombre{} est divisible par 4.
                        }{%
                        Comme \deuxdernierchiffres{} n'est pas divisible par 4 (reste \intcalcMod{\deuxdernierchiffres}{4}), \nombre{} n'est pas divisible par 4.
                    }%
                }%
                }{%
                \ifthenelse{\equal{\diviseur}{5}}{%
                    % Cas 5
                    \textbf{Critère de divisibilité par 5 :} Un nombre est divisible par 5 si son dernier chiffre est 0 ou 5.
                    
                    \textbf{Application à \nombre :} Le dernier chiffre de \nombre{} est \dernierchiffre.
                    
                    \ifthenelse{\equal{\intcalcMod{\nombre}{5}}{0}}{%
                        Comme le dernier chiffre est \dernierchiffre{} (qui est
                        \ifthenelse{\equal{\dernierchiffre}{0}}{0}{5}), \nombre{} est divisible par 5.
                        }{%
                        Comme le dernier chiffre est \dernierchiffre, qui n'est ni 0 ni 5, \nombre{} n'est pas divisible par 5.
                    }%
                    }{%
                    \ifthenelse{\equal{\diviseur}{6}}{%
                        % Cas 6
                        \textbf{Critère de divisibilité par 6 :} Un nombre est divisible par 6 s'il est divisible à la fois par 2 et par 3.
                        
                        \textbf{Application à \nombre :}
                        
                        \textbf{Divisibilité par 2 :} Un nombre est divisible par 2 si son dernier chiffre est pair.
                        
                        \ifthenelse{\equal{\intcalcMod{\nombre}{2}}{0}}{%
                            Le dernier chiffre de \nombre{} est \dernierchiffre, qui est pair, donc \nombre{} est divisible par 2.
                            }{%
                            Le dernier chiffre de \nombre{} est \dernierchiffre, qui n'est pas pair, donc \nombre{} n'est pas divisible par 2.
                        }%
                        
                        \textbf{Divisibilité par 3 :} Un nombre est divisible par 3 si la somme de ses chiffres est divisible par 3.
                        
                        \ifthenelse{\equal{\intcalcMod{\nombre}{3}}{0}}{%
                            La somme des chiffres de \nombre{} est divisible par 3, donc \nombre{} est divisible par 3.
                            }{%
                            La somme des chiffres de \nombre{} n'est pas divisible par 3, donc \nombre{} n'est pas divisible par 3.
                        }%
                        
                        \textbf{Conclusion :}
                        
                        \ifthenelse{\equal{\intcalcMod{\nombre}{6}}{0}}{%
                            Comme \nombre{} est divisible à la fois par 2 et par 3, \nombre{} est divisible par 6.
                            }{%
                            \ifthenelse{\equal{\intcalcMod{\nombre}{2}}{0}}{%
                                Comme \nombre{} n'est pas divisible par 3, \nombre{} n'est pas divisible par 6.
                                }{%
                                Comme \nombre{} n'est pas divisible par 2, \nombre{} n'est pas divisible par 6.
                            }%
                        }%
                        }{%
                        \ifthenelse{\equal{\diviseur}{9}}{%
                            % Cas 9
                            \textbf{Critère de divisibilité par 9 :} Un nombre est divisible par 9 si la somme de ses chiffres est divisible par 9.
                            
                            \textbf{Application à \nombre :}
                            
                            Pour vérifier si \nombre{} est divisible par 9, calculons la somme de ses chiffres.
                            
                            % Pour les besoins de l'exemple, nous affichons directement le résultat
                            \ifthenelse{\equal{\intcalcMod{\nombre}{9}}{0}}{%
                                La somme des chiffres de \nombre{} est divisible par 9, donc \nombre{} est divisible par 9.
                                }{%
                                La somme des chiffres de \nombre{} n'est pas divisible par 9, donc \nombre{} n'est pas divisible par 9.
                            }%
                            }{%
                            % Diviseur invalide
                            %\textbf{Erreur :} Le diviseur \diviseur{} n'est pas dans la liste autorisée (2, 3, 4, 5, 6, 9).
                            Aucun critère ne fonctionne.
                        }%
                    }%
                }%
            }%
        }%
    }%
}


\ExplSyntaxOn

% Définition de la commande \para
% Si \na > 0, retourne \na ; si \na < 0, retourne (\na)
\cs_new:Npn \para #1
{
    \int_compare:nNnTF {#1} > {0}
    {#1} % Si #1 est positif, retourne #1
    {(#1)} % Si #1 est négatif, retourne (#1)
}

% Définition de la commande \expa
% Si \puisxa = 0, retourne \para x^{\puisxa}, sinon retourne \para
\cs_new:Npn \expa #1 #2
{
    \int_compare:nNnTF {#2} = {0}
    {\para{#1}} % Si la puissance est 0, affiche avec x^{#2}
    {
        \int_compare:nNnTF {#2} = {1}
        {\para{#1}~x}
        {\para{#1}~x^{#2}}
    } % Sinon, retourne simplement \para
}
\cs_new:Npn \expan #1 #2
{
    \int_compare:nNnTF {#2} = {0}
    {#1} % Si la puissance est 0, affiche avec x^{#2}
    {
        \int_compare:nNnTF {#2} = {1}
        {#1~x}
        {#1~x^{#2}}
    } % Sinon, retourne simplement \para
}
\cs_new:Npn \finalsigne #1
{
    \int_compare:nNnTF {#1} > {0}
    {+} % Si la puissance est 0, affiche avec x^{#2}
    {-} % Sinon, retourne simplement \para
}
\ExplSyntaxOff
% Définition de la couleur personnalisée
\definecolor{customBackground}{RGB}{0, 43, 54}

\def\boiteQFdetcouleur{customBackground}%{green!75!black}

\newcommand\boiteQFdet[2]{
    \begin{tcolorbox}[nobeforeafter,title=#1,halign title=flush left,fonttitle=\bfseries,colbacktitle=\boiteQFdetcouleur,colframe=customBackground,coltitle=white,colback=white,left=0.2pt,right=0.2pt,width=17.5cm]
        #2
    \end{tcolorbox}
}
\newcommand{\calculatriceAutorisee}{}
% On définit un booléen pour la détection
\newif\ifcalcAllowed
\newcommand\boiteQFQ[2]{%
    % Vérifie si #2 contient la chaîne "\calculatriceAutorisee"
    \IfSubStr{\detokenize{#2}}{\string\calculatriceAutorisee}{%
        % Si trouvé : titre avec l'icône
        \calcAllowedtrue
        }{%
        \calcAllowedfalse
    }%
    %\def\mytitle{#1}
    \begin{tcolorbox}[%
        nobeforeafter,
        title=#1,
        halign title=flush left,
        fonttitle=\bfseries,
        colbacktitle=customBackground,
        colframe=customBackground,
        coltitle=white,
        colback=white,
        left=0.2pt,
        right=0.2pt,
        height=4cm,
        width=8.2cm,
        % L'overlay ne modifie pas la géométrie du titre
        overlay={%
            \ifcalcAllowed
            % Place l'image en overlay à la droite du titre (ajustez xshift/yshift si besoin)
        \node[anchor=base west] at ([xshift=2.5cm,yshift=-0.325cm]title.west) {%xshift=-0.93cm,yshift=-0.325cm]title.east
            \includegraphics[width=0.8cm,height=0.8cm]{C:/Users/Utilisateur/Desktop/Faire/Macros/Programmes_de_facilitation/QF_generator/images/je_calcule.png}%
        };
        \fi
    }%
]
%\begin{itembox}\textbf{#1}\end{itembox}
#2
\end{tcolorbox}
}
%\def\boitereponsecouleur{gray!65!red}
\def\boitereponsecouleur{red!75!black}
\newcommand\boiteQFA[1]{
\begin{tcolorbox}[nobeforeafter,title=Réponses :,halign title=center,fonttitle=\bfseries,colbacktitle=\boitereponsecouleur,colframe=customBackground,coltitle=white,colback=white,left=5pt,right=0.2pt,height=4cm,width=8.2cm]
#1
\end{tcolorbox}
}

\def\myiconpath{}

\getRandomIconPath

\renewcommand\boiteQFA[1]{
\begin{tcolorbox}[nobeforeafter, title=Réponses :, halign title= flush left,
fonttitle=\bfseries, colbacktitle=\boitereponsecouleur,
coltitle=white, colback=white,colframe=customBackground, left=5pt, right=0.2pt,
height=4cm, width=8.2cm, enhanced, overlay={
    \node[anchor=center] at (frame.north west) [xshift=-0.5cm, yshift=0cm]
    {
        %\includegraphics[width=1.5cm]{../../../images/sticker_eclair.png}
        \includegraphics[width=1.5cm]{\myiconpath }
        %\randomincludegraphics[2cm][0cm][0cm]
    };
}]
#1
\end{tcolorbox}
}
\newcommand\boiteQzeroFA[1]{
\begin{tcolorbox}[nobeforeafter, title=Réponses :, halign title= flush left,
fonttitle=\bfseries, colbacktitle=\boitereponsecouleur,
coltitle=white, colback=white,colframe=customBackground, left=5pt, right=0.2pt,
height=4cm, width=8.2cm, enhanced, overlay={
\node[anchor=center] at (frame.north west) [xshift=-0.5cm, yshift=0cm]
{
    \includegraphics[width=1.5cm]{C:/Users/Utilisateur/Desktop/Faire/Macros/Programmes_de_facilitation/QF_generator/images/sticker_eclair.png}%{../../../images/sticker_eclair.png}
    %\includegraphics[width=1.5cm]{\myiconpath }
    %\randomincludegraphics[2cm][0cm][0cm]
};
}]
\begin{center}
\includegraphics[height=2.8cm]{\myiconpath }%
\end{center}
\end{tcolorbox}
}
\newtcolorbox{itembox}[1][]{
baseline,
nobeforeafter, % Supprime les espaces avant et après
colframe=customBackground,
colback=customBackground,
coltext=white,
coltitle=white,
boxsep=1pt,
boxrule=0.5pt,
arc=2pt,
left=2pt,
right=2pt,
top=1pt,
bottom=1pt,
width=0.7cm,
#1
}
\usepackage{graphicx}
\usepackage{fontawesome5}
\usepackage{siunitx} % Pour la notation avec virgule
\usepackage{xfp}
% Configurer siunitx pour utiliser la virgule comme séparateur décimal
%\sisetup{output-decimal-marker={,}}
\newcommand{\cperso}[1]{\num{\fpeval{ #1 }}}
% Configuration locale pour le formatage des nombres
\sisetup{%
%  round-mode=places,
%  round-precision=5,
output-decimal-marker={,},
%  group-separator={}
}

% Définir la commande \response
\newcommand{\fracrep}[2]{%
\FPeval{\result}{round(#1/#2:8)} % Calculer la fraction et arrondir à 4 décimales
\num{\result} % Afficher le résultat avec une virgule
}
\opset{decimalsepsymbol={,}}
%\newcommand{\degree}{^\circ}
\begin{document}
%niveau estimé : 1ere
%date de projection : 01_10_2025
%themes abordés : Polynomes degré 2, Trigonométrie, Fractions
\pagecolor{customBackground}
\vspace{-0.5cm}
\begin{multicols}{2}
\boiteQFQ{Question 1 :}{
%version 7 - Court

\def\coeffa{-7}
\def\coeffb{-4}
\def\coeffc{-3}
\def\mode{2}




\def\myalpha{\fpeval{-\coeffb/(2*\coeffa)}}
\def\mybeta{\fpeval{(\coeffb*\coeffb-4*\coeffa*\coeffc)/(4*\coeffa)}}
\xdef\myalphafrac{\dfrac{\fpeval{-\coeffb}}{\fpeval{2*\coeffa}}}
\xdef\mybetafrac{\dfrac{\fpeval{\coeffb*\coeffb-4*\coeffa*\coeffc}}{\fpeval{4*\coeffa}}}
\xdef\myalphasimp{\displaystyle{\Simplification{\fpeval{-\coeffb}}{\fpeval{2*\coeffa}}}}
\xdef\mybetasimp{\displaystyle{\Simplification{\fpeval{\coeffb*\coeffb-4*\coeffa*\coeffc}}{\fpeval{4*\coeffa}}}}



\ifthenelse{\mode=1}{\acc{Développer} \begin{center}$\affiche{\coeffa}\left(x - \myalphasimp\right)^2 + \mybetasimp$\end{center}}{\acc{Déterminer la forme canonique de} \begin{center}$\affiche{\coeffa}x^2 + \affiche{\coeffb}x + \affiche{\coeffc}$\end{center}}
}
\boiteQFQ{Question 2 :}{
%version 1 - Court

\def\mfonction{sin}
\def\mden{3}
\def\mnum{-2}








\acc{Déterminer la valeur exacte de} $\text{\mfonction}\left(\ifnum\mnum=1\dfrac{\pi}{\mden}\else\ifnum\mnum=-1\dfrac{-\pi}{\mden}\else\dfrac{\mnum\pi}{\mden}\fi\fi\right)$
}
\end{multicols}
\vspace{-0.85cm}
\begin{multicols}{2}

\boiteQFQ{Question 3 :}{
%version 8 - Court

\def\n{11}
\def\d{12}
\def\nb{5}
\def\db{16}


Effectuer le calcul suivant en donnant le résultat sous forme simplifiée :     $$\dfrac{\n}{\d} + \dfrac{\nb}{\db}$$
}
\boiteQzeroFA{}
\end{multicols}

\newpage
\vspace{-0.5cm}
\begin{multicols}{2}
\boiteQFQ{Question 1 :}{
%version 7 - Court

\def\coeffa{-7}
\def\coeffb{-4}
\def\coeffc{-3}
\def\mode{2}




\def\myalpha{\fpeval{-\coeffb/(2*\coeffa)}}
\def\mybeta{\fpeval{(\coeffb*\coeffb-4*\coeffa*\coeffc)/(4*\coeffa)}}
\xdef\myalphafrac{\dfrac{\fpeval{-\coeffb}}{\fpeval{2*\coeffa}}}
\xdef\mybetafrac{\dfrac{\fpeval{\coeffb*\coeffb-4*\coeffa*\coeffc}}{\fpeval{4*\coeffa}}}
\xdef\myalphasimp{\displaystyle{\Simplification{\fpeval{-\coeffb}}{\fpeval{2*\coeffa}}}}
\xdef\mybetasimp{\displaystyle{\Simplification{\fpeval{\coeffb*\coeffb-4*\coeffa*\coeffc}}{\fpeval{4*\coeffa}}}}



\ifthenelse{\mode=1}{\acc{Développer} \begin{center}$\affiche{\coeffa}\left(x - \myalphasimp\right)^2 + \mybetasimp$\end{center}}{\acc{Déterminer la forme canonique de} \begin{center}$\affiche{\coeffa}x^2 + \affiche{\coeffb}x + \affiche{\coeffc}$\end{center}}
}
\boiteQFQ{Question 2 :}{
%version 1 - Court

\def\mfonction{sin}
\def\mden{3}
\def\mnum{-2}








\acc{Déterminer la valeur exacte de} $\text{\mfonction}\left(\ifnum\mnum=1\dfrac{\pi}{\mden}\else\ifnum\mnum=-1\dfrac{-\pi}{\mden}\else\dfrac{\mnum\pi}{\mden}\fi\fi\right)$
}

\end{multicols}
\vspace{-0.85cm}
\begin{multicols}{2}

\boiteQFQ{Question 3 :}{
%version 8 - Court

\def\n{11}
\def\d{12}
\def\nb{5}
\def\db{16}


Effectuer le calcul suivant en donnant le résultat sous forme simplifiée :     $$\dfrac{\n}{\d} + \dfrac{\nb}{\db}$$
}
\boiteQFA{
\vspace{-0.35cm}
\begin{itemize}[itemsep=0.2em]
\item[\raisebox{-0.1cm}{\begin{itembox} \textbf{1.} \end{itembox}}] \def\coeffa{-7}\def\coeffb{-4}\def\coeffc{-3}\def\mode{2}\def\myalpha{\fpeval{-\coeffb/(2*\coeffa)}}\def\mybeta{\fpeval{(\coeffb*\coeffb-4*\coeffa*\coeffc)/(4*\coeffa)}}\xdef\myalphafrac{\dfrac{\fpeval{-\coeffb}}{\fpeval{2*\coeffa}}}\xdef\mybetafrac{\dfrac{\fpeval{\coeffb*\coeffb-4*\coeffa*\coeffc}}{\fpeval{-4*\coeffa}}}\xdef\myalphasimp{\displaystyle{\Simplification{\fpeval{-\coeffb}}{\fpeval{2*\coeffa}}}}\xdef\mybetasimp{\displaystyle{\Simplification{\fpeval{\coeffb*\coeffb-4*\coeffa*\coeffc}}{\fpeval{-4*\coeffa}}}}\ifthenelse{\mode=1}{$\affiche{\coeffa}x^2 + \affiche{\coeffb}x + \affiche{\coeffc}$}{$\affiche{\coeffa}\left(x - \myalphasimp\right)^2 + \mybetasimp$}
\end{itemize}
}
\end{multicols}
\newpage
\vspace{-0.5cm}
\begin{multicols}{2}
\boiteQFQ{Question 1 :}{
%version 7 - Court

\def\coeffa{-7}
\def\coeffb{-4}
\def\coeffc{-3}
\def\mode{2}




\def\myalpha{\fpeval{-\coeffb/(2*\coeffa)}}
\def\mybeta{\fpeval{(\coeffb*\coeffb-4*\coeffa*\coeffc)/(4*\coeffa)}}
\xdef\myalphafrac{\dfrac{\fpeval{-\coeffb}}{\fpeval{2*\coeffa}}}
\xdef\mybetafrac{\dfrac{\fpeval{\coeffb*\coeffb-4*\coeffa*\coeffc}}{\fpeval{4*\coeffa}}}
\xdef\myalphasimp{\displaystyle{\Simplification{\fpeval{-\coeffb}}{\fpeval{2*\coeffa}}}}
\xdef\mybetasimp{\displaystyle{\Simplification{\fpeval{\coeffb*\coeffb-4*\coeffa*\coeffc}}{\fpeval{4*\coeffa}}}}



\ifthenelse{\mode=1}{\acc{Développer} \begin{center}$\affiche{\coeffa}\left(x - \myalphasimp\right)^2 + \mybetasimp$\end{center}}{\acc{Déterminer la forme canonique de} \begin{center}$\affiche{\coeffa}x^2 + \affiche{\coeffb}x + \affiche{\coeffc}$\end{center}}
}
\boiteQFQ{Question 2 :}{
%version 1 - Court

\def\mfonction{sin}
\def\mden{3}
\def\mnum{-2}








\acc{Déterminer la valeur exacte de} $\text{\mfonction}\left(\ifnum\mnum=1\dfrac{\pi}{\mden}\else\ifnum\mnum=-1\dfrac{-\pi}{\mden}\else\dfrac{\mnum\pi}{\mden}\fi\fi\right)$
}

\end{multicols}
\vspace{-0.85cm}
\begin{multicols}{2}

\boiteQFQ{Question 3 :}{
%version 8 - Court

\def\n{11}
\def\d{12}
\def\nb{5}
\def\db{16}


Effectuer le calcul suivant en donnant le résultat sous forme simplifiée :     $$\dfrac{\n}{\d} + \dfrac{\nb}{\db}$$
}
\boiteQFA{
\vspace{-0.35cm}
\begin{itemize}[itemsep=0.2em]
\item[\raisebox{-0.1cm}{\begin{itembox} \textbf{1.} \end{itembox}}] \def\coeffa{-7}\def\coeffb{-4}\def\coeffc{-3}\def\mode{2}\def\myalpha{\fpeval{-\coeffb/(2*\coeffa)}}\def\mybeta{\fpeval{(\coeffb*\coeffb-4*\coeffa*\coeffc)/(4*\coeffa)}}\xdef\myalphafrac{\dfrac{\fpeval{-\coeffb}}{\fpeval{2*\coeffa}}}\xdef\mybetafrac{\dfrac{\fpeval{\coeffb*\coeffb-4*\coeffa*\coeffc}}{\fpeval{-4*\coeffa}}}\xdef\myalphasimp{\displaystyle{\Simplification{\fpeval{-\coeffb}}{\fpeval{2*\coeffa}}}}\xdef\mybetasimp{\displaystyle{\Simplification{\fpeval{\coeffb*\coeffb-4*\coeffa*\coeffc}}{\fpeval{-4*\coeffa}}}}\ifthenelse{\mode=1}{$\affiche{\coeffa}x^2 + \affiche{\coeffb}x + \affiche{\coeffc}$}{$\affiche{\coeffa}\left(x - \myalphasimp\right)^2 + \mybetasimp$}
\item[\raisebox{-0.1cm}{\begin{itembox} \textbf{2.} \end{itembox}}] \def\mfonction{sin}
\def\mden{3}
\def\mnum{-2}\foreach \numang/\denang/\valcos/\valsin in {1/6/\dfrac{\sqrt{3}}{2}/\dfrac{1}{2},-1/6/-\dfrac{\sqrt{3}}{2}/-\dfrac{1}{2},2/6/\dfrac{1}{2}/\dfrac{\sqrt{3}}{2},-2/6/-\dfrac{1}{2}/-\dfrac{\sqrt{3}}{2},3/6/0/1,-3/6/0/-1,4/6/-\dfrac{1}{2}/\dfrac{\sqrt{3}}{2},-4/6/\dfrac{1}{2}/-\dfrac{\sqrt{3}}{2},5/6/-\dfrac{\sqrt{3}}{2}/\dfrac{1}{2},-5/6/\dfrac{\sqrt{3}}{2}/-\dfrac{1}{2},6/6/-1/0,-6/6/-1/0,1/4/\dfrac{\sqrt{2}}{2}/\dfrac{\sqrt{2}}{2},-1/4/\dfrac{\sqrt{2}}{2}/-\dfrac{\sqrt{2}}{2},2/4/0/1,-2/4/0/-1,3/4/-\dfrac{\sqrt{2}}{2}/\dfrac{\sqrt{2}}{2},-3/4/-\dfrac{\sqrt{2}}{2}/-\dfrac{\sqrt{2}}{2},4/4/-1/0,-4/4/-1/0,1/3/\dfrac{1}{2}/\dfrac{\sqrt{3}}{2},-1/3/\dfrac{1}{2}/-\dfrac{\sqrt{3}}{2},2/3/-\dfrac{1}{2}/\dfrac{\sqrt{3}}{2},-2/3/-\dfrac{1}{2}/-\dfrac{\sqrt{3}}{2},3/3/-1/0,-3/3/-1/0,1/2/0/1,-1/2/0/-1,2/2/-1/0,-2/2/-1/0}{\ifnum\numang=\mnum\ifnum\denang=\mden\IfStrEq{\mfonction}{cos}{$\text{\mfonction}\left(\ifnum\mnum=1\dfrac{\pi}{\mden}\else\ifnum\mnum=-1\dfrac{-\pi}{\mden}\else\dfrac{\mnum\pi}{\mden}\fi\fi\right) = \valcos$}{}\IfStrEq{\mfonction}{sin}{$\text{\mfonction}\left(\ifnum\mnum=1\dfrac{\pi}{\mden}\else\ifnum\mnum=-1\dfrac{-\pi}{\mden}\else\dfrac{\mnum\pi}{\mden}\fi\fi\right) = \valsin$}{}\fi\fi}
\end{itemize}
}
\end{multicols}
\newpage

\begin{multicols}{2}
\boiteQFQ{Question 1 :}{
%version 7 - Court

\def\coeffa{-7}
\def\coeffb{-4}
\def\coeffc{-3}
\def\mode{2}




\def\myalpha{\fpeval{-\coeffb/(2*\coeffa)}}
\def\mybeta{\fpeval{(\coeffb*\coeffb-4*\coeffa*\coeffc)/(4*\coeffa)}}
\xdef\myalphafrac{\dfrac{\fpeval{-\coeffb}}{\fpeval{2*\coeffa}}}
\xdef\mybetafrac{\dfrac{\fpeval{\coeffb*\coeffb-4*\coeffa*\coeffc}}{\fpeval{4*\coeffa}}}
\xdef\myalphasimp{\displaystyle{\Simplification{\fpeval{-\coeffb}}{\fpeval{2*\coeffa}}}}
\xdef\mybetasimp{\displaystyle{\Simplification{\fpeval{\coeffb*\coeffb-4*\coeffa*\coeffc}}{\fpeval{4*\coeffa}}}}



\ifthenelse{\mode=1}{\acc{Développer} \begin{center}$\affiche{\coeffa}\left(x - \myalphasimp\right)^2 + \mybetasimp$\end{center}}{\acc{Déterminer la forme canonique de} \begin{center}$\affiche{\coeffa}x^2 + \affiche{\coeffb}x + \affiche{\coeffc}$\end{center}}
}
\boiteQFQ{Question 2 :}{
%version 1 - Court

\def\mfonction{sin}
\def\mden{3}
\def\mnum{-2}








\acc{Déterminer la valeur exacte de} $\text{\mfonction}\left(\ifnum\mnum=1\dfrac{\pi}{\mden}\else\ifnum\mnum=-1\dfrac{-\pi}{\mden}\else\dfrac{\mnum\pi}{\mden}\fi\fi\right)$
}

\end{multicols}
\vspace{-0.85cm}
\begin{multicols}{2}

\boiteQFQ{Question 3 :}{
%version 8 - Court

\def\n{11}
\def\d{12}
\def\nb{5}
\def\db{16}


Effectuer le calcul suivant en donnant le résultat sous forme simplifiée :     $$\dfrac{\n}{\d} + \dfrac{\nb}{\db}$$
}
\boiteQFA{
\vspace{-0.35cm}
\begin{itemize}[itemsep=0.2em]
\item[\raisebox{-0.1cm}{\begin{itembox} \textbf{1.} \end{itembox}}] \def\coeffa{-7}\def\coeffb{-4}\def\coeffc{-3}\def\mode{2}\def\myalpha{\fpeval{-\coeffb/(2*\coeffa)}}\def\mybeta{\fpeval{(\coeffb*\coeffb-4*\coeffa*\coeffc)/(4*\coeffa)}}\xdef\myalphafrac{\dfrac{\fpeval{-\coeffb}}{\fpeval{2*\coeffa}}}\xdef\mybetafrac{\dfrac{\fpeval{\coeffb*\coeffb-4*\coeffa*\coeffc}}{\fpeval{-4*\coeffa}}}\xdef\myalphasimp{\displaystyle{\Simplification{\fpeval{-\coeffb}}{\fpeval{2*\coeffa}}}}\xdef\mybetasimp{\displaystyle{\Simplification{\fpeval{\coeffb*\coeffb-4*\coeffa*\coeffc}}{\fpeval{-4*\coeffa}}}}\ifthenelse{\mode=1}{$\affiche{\coeffa}x^2 + \affiche{\coeffb}x + \affiche{\coeffc}$}{$\affiche{\coeffa}\left(x - \myalphasimp\right)^2 + \mybetasimp$}
\item[\raisebox{-0.1cm}{\begin{itembox} \textbf{2.} \end{itembox}}] \def\mfonction{sin}
\def\mden{3}
\def\mnum{-2}\foreach \numang/\denang/\valcos/\valsin in {1/6/\dfrac{\sqrt{3}}{2}/\dfrac{1}{2},-1/6/-\dfrac{\sqrt{3}}{2}/-\dfrac{1}{2},2/6/\dfrac{1}{2}/\dfrac{\sqrt{3}}{2},-2/6/-\dfrac{1}{2}/-\dfrac{\sqrt{3}}{2},3/6/0/1,-3/6/0/-1,4/6/-\dfrac{1}{2}/\dfrac{\sqrt{3}}{2},-4/6/\dfrac{1}{2}/-\dfrac{\sqrt{3}}{2},5/6/-\dfrac{\sqrt{3}}{2}/\dfrac{1}{2},-5/6/\dfrac{\sqrt{3}}{2}/-\dfrac{1}{2},6/6/-1/0,-6/6/-1/0,1/4/\dfrac{\sqrt{2}}{2}/\dfrac{\sqrt{2}}{2},-1/4/\dfrac{\sqrt{2}}{2}/-\dfrac{\sqrt{2}}{2},2/4/0/1,-2/4/0/-1,3/4/-\dfrac{\sqrt{2}}{2}/\dfrac{\sqrt{2}}{2},-3/4/-\dfrac{\sqrt{2}}{2}/-\dfrac{\sqrt{2}}{2},4/4/-1/0,-4/4/-1/0,1/3/\dfrac{1}{2}/\dfrac{\sqrt{3}}{2},-1/3/\dfrac{1}{2}/-\dfrac{\sqrt{3}}{2},2/3/-\dfrac{1}{2}/\dfrac{\sqrt{3}}{2},-2/3/-\dfrac{1}{2}/-\dfrac{\sqrt{3}}{2},3/3/-1/0,-3/3/-1/0,1/2/0/1,-1/2/0/-1,2/2/-1/0,-2/2/-1/0}{\ifnum\numang=\mnum\ifnum\denang=\mden\IfStrEq{\mfonction}{cos}{$\text{\mfonction}\left(\ifnum\mnum=1\dfrac{\pi}{\mden}\else\ifnum\mnum=-1\dfrac{-\pi}{\mden}\else\dfrac{\mnum\pi}{\mden}\fi\fi\right) = \valcos$}{}\IfStrEq{\mfonction}{sin}{$\text{\mfonction}\left(\ifnum\mnum=1\dfrac{\pi}{\mden}\else\ifnum\mnum=-1\dfrac{-\pi}{\mden}\else\dfrac{\mnum\pi}{\mden}\fi\fi\right) = \valsin$}{}\fi\fi}
\item[\raisebox{-0.1cm}{\begin{itembox} \textbf{3.} \end{itembox}}] \def\n{11}
\def\d{12}
\def\nb{5}
\def\db{16}\def\nresa{\fpeval{\n + \nb}}    \def\nresb{\fpeval{\n * \db + \nb * \d}}    \def\nresc{\fpeval{\n + \nb * \d / \db}}    \def\nresd{\fpeval{\n * \db + \nb * \d}}    \def\dresa{\fpeval{\d + \db}}    \def\dresb{\fpeval{\d * \db}}    \def\dresc{\fpeval{\d}}    \def\dresd{\fpeval{\d + \db}}    $\displaystyle\Simplification{\nresb}{\dresb}$
\end{itemize}
}
\end{multicols}

\newpage

%\begin{multicols}{2}
\boiteQFdet{Solution détaillée de la question 1 :}{

%version 7 - Court

\def\coeffa{-7}
\def\coeffb{-4}
\def\coeffc{-3}
\def\mode{2}




\def\myalpha{\fpeval{-\coeffb/(2*\coeffa)}}
\def\mybeta{\fpeval{(\coeffb*\coeffb-4*\coeffa*\coeffc)/(4*\coeffa)}}
\xdef\myalphafrac{\dfrac{\fpeval{-\coeffb}}{\fpeval{2*\coeffa}}}
\xdef\mybetafrac{\dfrac{\fpeval{\coeffb*\coeffb-4*\coeffa*\coeffc}}{\fpeval{4*\coeffa}}}
\xdef\myalphasimp{\displaystyle{\Simplification{\fpeval{-\coeffb}}{\fpeval{2*\coeffa}}}}
\xdef\mybetasimp{\displaystyle{\Simplification{\fpeval{\coeffb*\coeffb-4*\coeffa*\coeffc}}{\fpeval{4*\coeffa}}}}



\ifthenelse{\mode=1}{\acc{Développer} \begin{center}$\affiche{\coeffa}\left(x - \myalphasimp\right)^2 + \mybetasimp$\end{center}}{\acc{Déterminer la forme canonique de} \begin{center}$\affiche{\coeffa}x^2 + \affiche{\coeffb}x + \affiche{\coeffc}$\end{center}}

\tikz{\draw[dashed, line width=1pt] (0,0) -- (\linewidth,0);}

\vspace{-0.25cm}\setlength{\columnseprule}{0.4pt}\begin{multicols}{2}


%version 7 - solutions

\def\coeffa{-7}
\def\coeffb{-4}
\def\coeffc{-3}
\def\mode{2}




\def\myalpha{\fpeval{-\coeffb/(2*\coeffa)}}
\def\mybeta{\fpeval{(\coeffb*\coeffb-4*\coeffa*\coeffc)/(-4*\coeffa)}}
\def\mybetanum{\fpeval{\coeffb*\coeffb-4*\coeffa*\coeffc}}
\def\mybetaden{\fpeval{-4*\coeffa}}
\xdef\myalphafrac{\dfrac{\fpeval{-\coeffb}}{\fpeval{2*\coeffa}}}
\xdef\mybetafrac{\dfrac{\fpeval{\coeffb*\coeffb-4*\coeffa*\coeffc}}{\fpeval{-4*\coeffa}}}
\xdef\myalphasimp{\displaystyle{\Simplification{\fpeval{-\coeffb}}{\fpeval{2*\coeffa}}}}
\xdef\mybetasimp{\displaystyle{\Simplification{\fpeval{\coeffb*\coeffb-4*\coeffa*\coeffc}}{\fpeval{-4*\coeffa}}}}

\ifthenelse{\mode=1}{
\textbf{Rappel :} $(x - \alpha)^2 = x^2 - 2\alpha x + \alpha^2$

Développons :
\begin{align*}
&\affiche{\coeffa}(x - \myalphasimp)^2 + \mybetasimp\\
&= \affiche{\coeffa}\left[x^2 - 2 \times \myalphasimp \times x + \left(\myalphasimp\right)^2\right] + \mybetasimp\\
&\approx \affiche{\coeffa}x^2 - \affiche{\fpeval{round(\coeffa*2*\myalpha,0)}}x + \affiche{\fpeval{round(\coeffa*\myalpha^2,2)}} + \affiche{\fpeval{round(\mybetanum/\mybetaden,2)}} \\
&= \affiche{\coeffa}x^2 + \affiche{\coeffb}x + \affiche{\coeffc}
\end{align*}
}{
\textbf{Détermination de la forme canonique :}

Pour $ax^2 + bx + c$, la forme canonique est $a(x - \alpha)^2 + \beta$ avec :
\begin{multicols}{2}
\begin{itemize}
    \item $\alpha = -\dfrac{b}{2a}$
    \item $\beta = -\dfrac{b^2 - 4ac}{4a}$
\end{itemize}
\columnbreak

Avec $a = \affiche{\coeffa}$,

$b = \affiche{\coeffb}$ et $c = \affiche{\coeffc}$ :
\end{multicols}

\columnbreak

$\alpha = -\dfrac{\affiche{\coeffb}}{2 \times \affiche{\coeffa}} = \myalphafrac =$ \encadrer[defi]{$\myalphasimp$}

$\beta = -\dfrac{\affiche{\coeffb}^2 - 4 \times \affiche{\coeffa} \times \affiche{\coeffc}}{4 \times \affiche{\coeffa}} = -\dfrac{\affiche{\fpeval{\coeffb*\coeffb}} - \affiche{\fpeval{4*\coeffa*\coeffc}}}{\affiche{\fpeval{4*\coeffa}}}$

$\phantom{\beta } = \mybetafrac = $\encadrer[defi]{$\mybetasimp$}

Donc :

\begin{center}
\encadrer[defi]{$\affiche{\coeffa}x^2 + \affiche{\coeffb}x + \affiche{\coeffc} = \affiche{\coeffa}\left(x - \myalphasimp\right)^2 + \mybetasimp$}
\end{center}
}

\end{multicols}
}

\newpage

\boiteQFdet{Solution détaillée de la question 2 :}{

%version 1 - Court

\def\mfonction{sin}
\def\mden{3}
\def\mnum{-2}








\acc{Déterminer la valeur exacte de} $\text{\mfonction}\left(\ifnum\mnum=1\dfrac{\pi}{\mden}\else\ifnum\mnum=-1\dfrac{-\pi}{\mden}\else\dfrac{\mnum\pi}{\mden}\fi\fi\right)$

\tikz{\draw[dashed, line width=1pt] (0,0) -- (\linewidth,0);}


\vspace{-0.25cm}\setlength{\columnseprule}{0.4pt}\begin{multicols}{2}

%version 1 - solutions

\def\mfonction{sin}
\def\mden{3}
\def\mnum{-2}





Pour déterminer $\text{\mfonction}\left(\ifnum\mnum=1\dfrac{\pi}{\mden}\else\ifnum\mnum=-1\dfrac{-\pi}{\mden}\else\dfrac{\mnum\pi}{\mden}\fi\fi\right)$, on place le point $M$ associé à une rotation de $\alpha =\ifnum\mnum=1\dfrac{\pi}{\mden}\else\ifnum\mnum=-1\dfrac{-\pi}{\mden}\else\dfrac{\mnum\pi}{\mden}\fi\fi$ sur le cercle trigonométrique.
\def\mapropreliste{1/6/\dfrac{\sqrt{3}}{2}/\dfrac{1}{2},
-1/6/-\dfrac{\sqrt{3}}{2}/-\dfrac{1}{2},
2/6/\dfrac{1}{2}/\dfrac{\sqrt{3}}{2},
-2/6/-\dfrac{1}{2}/-\dfrac{\sqrt{3}}{2},
3/6/0/1,
-3/6/0/-1,
4/6/-\dfrac{1}{2}/\dfrac{\sqrt{3}}{2},
-4/6/\dfrac{1}{2}/-\dfrac{\sqrt{3}}{2},
5/6/-\dfrac{\sqrt{3}}{2}/\dfrac{1}{2},
-5/6/\dfrac{\sqrt{3}}{2}/-\dfrac{1}{2},
6/6/-1/0,
-6/6/-1/0,
1/4/\dfrac{\sqrt{2}}{2}/\dfrac{\sqrt{2}}{2},
-1/4/\dfrac{\sqrt{2}}{2}/-\dfrac{\sqrt{2}}{2},
2/4/0/1,
-2/4/0/-1,
3/4/-\dfrac{\sqrt{2}}{2}/\dfrac{\sqrt{2}}{2},
-3/4/-\dfrac{\sqrt{2}}{2}/-\dfrac{\sqrt{2}}{2},
4/4/-1/0,
-4/4/-1/0,
1/3/\dfrac{1}{2}/\dfrac{\sqrt{3}}{2},
-1/3/\dfrac{1}{2}/-\dfrac{\sqrt{3}}{2},
2/3/-\dfrac{1}{2}/\dfrac{\sqrt{3}}{2},
-2/3/-\dfrac{1}{2}/-\dfrac{\sqrt{3}}{2},
3/3/-1/0,
-3/3/-1/0,
1/2/0/1,
-1/2/0/-1,
2/2/-1/0,
-2/2/-1/0}

Le \IfStrEq{\mfonction}{cos}{\encadrer[defi]{cosinus}}{\encadrer[defi]{sinus}} de l'angle $\alpha=\ifnum\mnum=1\dfrac{\pi}{\mden}\else\ifnum\mnum=-1\dfrac{-\pi}{\mden}\else\dfrac{\mnum\pi}{\mden}\fi\fi$ correspond à \IfStrEq{\mfonction}{cos}{\encadrer[defi]{l'abscisse}}{\encadrer[defi]{l'ordonnée}} du point $M$.

Ainsi :
\foreach \numang/\denang/\valcos/\valsin in \mapropreliste {
\ifnum\numang=\mnum
\ifnum\denang=\mden
\IfStrEq{\mfonction}{cos}{%
\begin{center}\encadrer[defi]{$\text{\mfonction}\left(\ifnum\mnum=1\dfrac{\pi}{\mden}\else\ifnum\mnum=-1\dfrac{-\pi}{\mden}\else\dfrac{\mnum\pi}{\mden}\fi\fi\right) = \valcos$}\end{center}
}{}
\IfStrEq{\mfonction}{sin}{%
\begin{center}\encadrer[defi]{$\text{\mfonction}\left(\ifnum\mnum=1\dfrac{\pi}{\mden}\else\ifnum\mnum=-1\dfrac{-\pi}{\mden}\else\dfrac{\mnum\pi}{\mden}\fi\fi\right) = \valsin$}\end{center}
}{}
\fi
\fi
}


\columnbreak

\begin{center}
\begin{tikzpicture}[scale=1.5,cap=round,>=latex]
    % draw the coordinates
    %\draw[->] (-1.2cm,0cm) -- (1.2cm,0cm) node[right,fill=white] {$x$};
    %\draw[->] (0cm,-1.2cm) -- (0cm,1.2cm) node[above,fill=white] {$y$};
    % draw symmetries (axes and center) in dark green
    \draw[defi,thick] (-1.2cm,0cm) -- (1.2cm,0cm); % axe horizontal
    \draw[defi,thick] (0cm,-1.2cm) -- (0cm,1.2cm); % axe vertical
    \filldraw[defi] (0,0) circle(1.5pt); % centre O
    % draw the unit circle
    \draw[thick] (0cm,0cm) circle(1cm);
    % draw direct rotation (sens trigonométrique)
    %\draw[->,black,thin] (30:1.8) arc (30:60:1.15) node[midway,right] {\scriptsize $+$};
    % draw key angles
    \foreach \x in {0,30,45,60,90,120,135,150,180,210,225,240,270,300,315,330} {
        \draw[gray,thin] (0cm,0cm) -- (\x:1cm);
    }
    % draw the angle
    \draw[red,thick] (0,0) -- ({180*\mnum/\mden}:1cm);
    % highlight the points
    \filldraw[red] ({180*\mnum/\mden}:1cm) circle(1.2pt);
    % draw dotted lines to show coordinates
    \draw[dotted,red,line width=0.8pt] ({180*\mnum/\mden}:1cm) -- ({cos(180*\mnum/\mden)},0);
    \draw[dotted,red,line width=0.8pt] ({180*\mnum/\mden}:1cm) -- (0,{sin(180*\mnum/\mden)});
    % Points images par symétries en vert foncé
    % Symétrie axiale par rapport à l'axe x : (x, -y)
    \filldraw[defi] ({cos(180*\mnum/\mden)},{-sin(180*\mnum/\mden)}) circle(1.2pt);
    \draw[dotted,defi,line width=0.8pt] ({cos(180*\mnum/\mden)},{-sin(180*\mnum/\mden)}) -- ({cos(180*\mnum/\mden)},0);
    \draw[dotted,defi,line width=0.8pt] ({cos(180*\mnum/\mden)},{-sin(180*\mnum/\mden)}) -- (0,{-sin(180*\mnum/\mden)});
    % Symétrie axiale par rapport à l'axe y : (-x, y)
    \filldraw[defi] ({-cos(180*\mnum/\mden)},{sin(180*\mnum/\mden)}) circle(1.2pt);
    \draw[dotted,defi,line width=0.8pt] ({-cos(180*\mnum/\mden)},{sin(180*\mnum/\mden)}) -- ({-cos(180*\mnum/\mden)},0);
    \draw[dotted,defi,line width=0.8pt] ({-cos(180*\mnum/\mden)},{sin(180*\mnum/\mden)}) -- (0,{sin(180*\mnum/\mden)});
    % Symétrie centrale par rapport à O : (-x, -y)
    \filldraw[defi] ({-cos(180*\mnum/\mden)},{-sin(180*\mnum/\mden)}) circle(1.2pt);
    \draw[dotted,defi,line width=0.8pt] ({-cos(180*\mnum/\mden)},{-sin(180*\mnum/\mden)}) -- ({-cos(180*\mnum/\mden)},0);
    \draw[dotted,defi,line width=0.8pt] ({-cos(180*\mnum/\mden)},{-sin(180*\mnum/\mden)}) -- (0,{-sin(180*\mnum/\mden)});
    % Calcul de l'angle et détection du cadrant
    \pgfmathsetmacro{\anglefinal}{180*\mnum/\mden}
    \pgfmathsetmacro{\angleabs}{abs(\anglefinal)}
    \pgfmathsetmacro{\radius}{1.1}
    
    % Déterminer le cadrant et les ancres appropriées
    % Cadrant 1: 0° à 90° (supérieur droit) -> anchor=south west
    % Cadrant 2: 90° à 180° (supérieur gauche) -> anchor=south east
    % Cadrant 3: -180° à -90° ou 180° à 270° (inférieur gauche) -> anchor=north east
    % Cadrant 4: -90° à 0° ou 270° à 360° (inférieur droit) -> anchor=north west
    
    \pgfmathsetmacro{\cadrant}{
        (\anglefinal >= 0 && \anglefinal <= 90) ? 1 :
        (\anglefinal > 90 && \anglefinal <= 180) ? 2 :
        (\anglefinal < 0 && \anglefinal >= -90) ? 4 :
        3
    }
    
    % Affichage de l'angle avec position adaptée et graduation
    \pgfmathsetmacro{\absnum}{abs(\mnum)}
    \ifdim\anglefinal pt<0pt
    \ifnum\mnum=-1
    \ifdim\anglefinal pt>-90pt
    \draw[->,prop,thick] (\radius,0) arc (0:\anglefinal:\radius) node[midway,below right] {\normalsize $\alpha=\dfrac{-\pi}{\mden}$};
    \else
    \draw[->,prop,thick] (\radius,0) arc (0:\anglefinal:\radius) node[midway,below left] {\normalsize $\alpha=\dfrac{-\pi}{\mden}$};
    \fi
    \else
    \ifdim\anglefinal pt>-90pt
    \draw[->,prop,thick] (\radius,0) arc (0:\anglefinal:\radius) node[midway,below right] {\normalsize $\alpha=\dfrac{\mnum\pi}{\mden}$};
    \else
    \draw[->,prop,thick] (\radius,0) arc (0:\anglefinal:\radius) node[midway,below left] {\normalsize $\alpha=\dfrac{\mnum\pi}{\mden}$};
    \fi
    \fi
    % Graduation de l'angle (angles négatifs)
    \foreach \i in {1,...,\absnum} {
        \pgfmathsetmacro{\tickangle}{-\i*180/\mden}
        \draw[prop] (\tickangle:\radius-0.05) -- (\tickangle:\radius+0.05);
    }
    \else
    \ifnum\mnum=1
    \ifdim\anglefinal pt<90pt
    \draw[->,prop,thick] (\radius,0) arc (0:\anglefinal:\radius) node[midway,above right] {\normalsize $\alpha=\dfrac{\pi}{\mden}$};
    \else
    \draw[->,prop,thick] (\radius,0) arc (0:\anglefinal:\radius) node[midway,above left] {\normalsize $\alpha=\dfrac{\pi}{\mden}$};
    \fi
    \else
    \ifdim\anglefinal pt<90pt
    \draw[->,prop,thick] (\radius,0) arc (0:\anglefinal:\radius) node[midway,above right] {\normalsize $\alpha=\dfrac{\mnum\pi}{\mden}$};
    \else
    \draw[->,prop,thick] (\radius,0) arc (0:\anglefinal:\radius) node[midway,above left] {\normalsize $\alpha=\dfrac{\mnum\pi}{\mden}$};
    \fi
    \fi
    % Graduation de l'angle (angles positifs)
    \foreach \i in {1,...,\mnum} {
        \pgfmathsetmacro{\tickangle}{\i*180/\mden}
        \draw[prop] (\tickangle:\radius-0.05) -- (\tickangle:\radius+0.05);
    }
    \fi
    
    % Affichage des coordonnées depuis \mapropreliste avec position adaptée au cadrant
    \foreach \numang/\denang/\valcos/\valsin in \mapropreliste {
        \ifnum\numang=\mnum
        \ifnum\denang=\mden
        \pgfmathsetmacro{\myangle}{180*\mnum/\mden}
        % Cadrant 1: 0° à 90° (supérieur droit)
        \ifdim\myangle pt<0pt
        \ifdim\myangle pt>-90pt
        % Cadrant 4: -90° à 0° (inférieur droit)
        \node[red,anchor=north west] at ({180*\mnum/\mden}:1.2cm) {\scriptsize $M\left(\valcos~;~\valsin\right)$};
        \else
        % Cadrant 3: -180° à -90° (inférieur gauche)
        \node[red,anchor=north east] at ({180*\mnum/\mden}:1.2cm) {\scriptsize $M\left(\valcos~;~\valsin\right)$};
        \fi
        \else
        \ifdim\myangle pt>90pt
        % Cadrant 2: 90° à 180° (supérieur gauche)
        \node[red,anchor=south east] at ({180*\mnum/\mden}:1.2cm) {\scriptsize $M\left(\valcos~;~\valsin\right)$};
        \else
        % Cadrant 1: 0° à 90° (supérieur droit)
        \node[red,anchor=south west] at ({180*\mnum/\mden}:1.2cm) {\scriptsize $M\left(\valcos~;~\valsin\right)$};
        \fi
        \fi
        \fi
        \fi
    }
\end{tikzpicture}
\end{center}

\end{multicols}
}

%\columnbreak
\newpage


\boiteQFdet{Solution détaillée de la question 3 :}{

%version 8 - Court

\def\n{11}
\def\d{12}
\def\nb{5}
\def\db{16}


Effectuer le calcul suivant en donnant le résultat sous forme simplifiée :     $$\dfrac{\n}{\d} + \dfrac{\nb}{\db}$$

\tikz{\draw[dashed, line width=1pt] (0,0) -- (\linewidth,0);}

\vspace{-0.25cm}\setlength{\columnseprule}{0.4pt}\begin{multicols}{2}

%version 8 - solutions

\def\n{11}
\def\d{12}
\def\nb{5}
\def\db{16}


\def\decnum{\Decomposition[Longue]{\fpeval{\n * \db + \nb * \d}}}    \def\decden{\Decomposition[Longue]{\fpeval{\d * \db}}}    \begin{align*}        \dfrac{\n}{\d} + \dfrac{\nb}{\db}         &=~ \dfrac{\n \times \db}{\d \times \db} + \dfrac{\nb \times \d}{\db \times \d}\\        &=~ \dfrac{\fpeval{\n * \db + \nb * \d}}{\fpeval{\d * \db}}\\        &=~ \dfrac{\decnum}{\decden}\\        &=~ \Simplification{\fpeval{\n * \db + \nb * \d}}{\fpeval{\d * \db}}    \end{align*}

\end{multicols}
}
%\end{multicols}
\end{document}
