
\documentclass[12pt]{article}
\usepackage{geometry}

\geometry{paperwidth=18cm, paperheight=8.7cm, margin=0.2cm}

\usepackage{bfcours}
\usepackage{bfcours-fonts}
\usepackage{pagecolor} % Pour colorer le fond de la page
%\usetikzlibrary{backgrounds}
\usepackage{expl3} % Nécessaire pour utiliser le langage LaTeX3
\ExplSyntaxOn
% Définition de la commande \mprefix
\cs_new:Npn \mprefix #1
  {
    \int_case:nnF {#1}
      {
        {-12} { \text{pico} }
        {-9}  { \text{nano} }
        {-6}  { \text{micro} }
        {-3}  { \text{milli} }
        {-2}  { \text{centi} }
        {-1}  { \text{deci} }
        {0}   { \text{} } % Aucun préfixe pour la puissance 0
        {1}   { \text{deca} }
        {2}   { \text{hecto} }
        {3}   { \text{kilo} }
        {6}   { \text{mega} }
        {9}   { \text{giga} }
        {12}  { \text{tera} }
      }
      { \text{Valeur inconnue} } % Gestion des cas non pris en charge
  }
\ExplSyntaxOff

\ExplSyntaxOn

% Définition de la commande \para
% Si \na > 0, retourne \na ; si \na < 0, retourne (\na)
\cs_new:Npn \para #1
  {
    \int_compare:nNnTF {#1} > {0}
      {#1} % Si #1 est positif, retourne #1
      {(#1)} % Si #1 est négatif, retourne (#1)
  }

% Définition de la commande \expa
% Si \puisxa = 0, retourne \para x^{\puisxa}, sinon retourne \para
\cs_new:Npn \expa #1 #2
  {
    \int_compare:nNnTF {#2} = {0}
      {\para{#1}} % Si la puissance est 0, affiche avec x^{#2}
      {
        \int_compare:nNnTF {#2} = {1}
        {\para{#1}~x}
        {\para{#1}~x^{#2}}
      } % Sinon, retourne simplement \para
  }
\cs_new:Npn \finalsigne #1
  {
    \int_compare:nNnTF {#1} > {0}
      {+} % Si la puissance est 0, affiche avec x^{#2}
      {-} % Sinon, retourne simplement \para
  }
\ExplSyntaxOff
% Définition de la couleur personnalisée
\definecolor{customBackground}{RGB}{0, 43, 54}

\def\boiteQFdetcouleur{customBackground}%{green!75!black}

\newcommand\boiteQFdet[2]{
    \begin{tcolorbox}[nobeforeafter,title=#1,halign title=flush left,fonttitle=\bfseries,colbacktitle=\boiteQFdetcouleur,colframe=customBackground,coltitle=white,colback=white,left=0.2pt,right=0.2pt,width=17.5cm]
        #2
    \end{tcolorbox}
}
\def\boiteQFcouleur{blue!75!black}

\newcommand\boiteQFQ[2]{
    \begin{tcolorbox}[
            nobeforeafter,
            title=#1,
            halign title=flush left,
            fonttitle=\bfseries,
            colbacktitle=customBackground,
            colframe=customBackground,
            coltitle=white,
            colback=white,
            left=0.2pt,
            right=0.2pt,
            height=4cm,
            width=8.2cm
        ]
        %\begin{itembox}\textbf{#1}\end{itembox} 
        #2
    \end{tcolorbox}
}
%\def\boitereponsecouleur{gray!65!red}
\def\boitereponsecouleur{red!75!black}
\newcommand\boiteQFA[1]{
    \begin{tcolorbox}[nobeforeafter,title=Réponses :,halign title=center,fonttitle=\bfseries,colbacktitle=\boitereponsecouleur,colframe=customBackground,coltitle=white,colback=white,left=5pt,right=0.2pt,height=4cm,width=8.2cm]
        #1
    \end{tcolorbox}
}



%%%%%%%%%%%%%%%%%%%%%
\def\myiconpath{sticker_eclair.png}
%Pour choisir une image dans un repertoire donné : 
%demute les instructions suivantes
%\renewcommand{\dbiconpath}{C:/Users/Utilisateur/Documents/Icones}
%\newcommand{\coverAnime}{C:/Users/Utilisateur/Documents/Icones/Fun/covers_manga}
% Définition du chemin de base pour les images
%\newcommand{\imdbPath}{C:/Users/Utilisateur/Documents/Icones}
%\getRandomIconPath



%%%%%%%%%%%%%%%%%%%%%
\renewcommand\boiteQFA[1]{
    \begin{tcolorbox}[nobeforeafter, title=Réponses :, halign title= flush left, 
                      fonttitle=\bfseries, colbacktitle=\boitereponsecouleur, 
                      coltitle=white, colback=white,colframe=customBackground, left=5pt, right=0.2pt,
                      height=4cm, width=8.2cm, enhanced, overlay={
        \node[anchor=center] at (frame.north west) [xshift=-0.5cm, yshift=0cm] 
            {
                %\includegraphics[width=1.5cm]{../../../images/sticker_eclair.png}
                \includegraphics[width=1.5cm]{\myiconpath }
                %\randomincludegraphics[2cm][0cm][0cm]
            };
    }]
        #1
    \end{tcolorbox}
}

\newtcolorbox{itembox}[1][]{
  baseline,
  nobeforeafter, % Supprime les espaces avant et après
  colframe=customBackground,
  colback=customBackground,
  coltext=white,
  coltitle=white,
  boxsep=1pt,
  boxrule=0.5pt,
  arc=2pt,
  left=2pt,
  right=2pt,
  top=1pt,
  bottom=1pt,
  width=0.7cm,
  #1
}
\usepackage{graphicx}
\usepackage{fontawesome5}
\usepackage{siunitx} % Pour la notation avec virgule
\usepackage{xfp}
% Configurer siunitx pour utiliser la virgule comme séparateur décimal
%\sisetup{output-decimal-marker={,}}
\newcommand{\cperso}[1]{\num{\fpeval{ #1 }}}
% Configuration locale pour le formatage des nombres
\sisetup{%
%  round-mode=places,
%  round-precision=5,
  output-decimal-marker={,},
%  group-separator={}
}

% Définir la commande \response
\newcommand{\fracrep}[2]{%
  \FPeval{\result}{round(#1/#2:8)} % Calculer la fraction et arrondir à 4 décimales
  \num{\result} % Afficher le résultat avec une virgule
}
%\newcommand{\degree}{^\circ}
\begin{document}
%metadata
\pagecolor{customBackground}
\vspace{-0.5cm}
\begin{multicols}{2}
    \boiteQFQ{Question 1 :}{
          %question1
    }
    \boiteQFQ{Question 2 :}{
          %question2
    }
\end{multicols}   
\vspace{-0.85cm}
\begin{multicols}{2}

    \boiteQFQ{Question 3 :}{
        %question3
    }       
    \boiteQFA{}
  \end{multicols}   

    \newpage
\vspace{-0.5cm}
\begin{multicols}{2}
    \boiteQFQ{Question 1 :}{
          %question1
    }
    \boiteQFQ{Question 2 :}{
          %question2
    }
          
\end{multicols}   
\vspace{-0.85cm}
\begin{multicols}{2}
  
    \boiteQFQ{Question 3 :}{
        %question3
    }
    \boiteQFA{
       \vspace{-0.35cm} 
        \begin{itemize}[itemsep=0.2em]
            \item[\raisebox{-0.1cm}{\begin{itembox} \textbf{1.} \end{itembox}}] %reponse1    
        \end{itemize}
    }
\end{multicols}   
    \newpage
\vspace{-0.5cm}
\begin{multicols}{2}
    \boiteQFQ{Question 1 :}{
        %question1
    }
    \boiteQFQ{Question 2 :}{
        %question2
    }
     
\end{multicols}   
\vspace{-0.85cm}
\begin{multicols}{2}
   
    \boiteQFQ{Question 3 :}{
        %question3
    }
    \boiteQFA{
        \vspace{-0.35cm} 
        \begin{itemize}[itemsep=0.2em]
            \item[\raisebox{-0.1cm}{\begin{itembox} \textbf{1.} \end{itembox}}] %reponse1    
            \item[\raisebox{-0.1cm}{\begin{itembox} \textbf{2.} \end{itembox}}] %reponse2
        \end{itemize}
    }
\end{multicols}   
    \newpage

\begin{multicols}{2}
    \boiteQFQ{Question 1 :}{
      %question1
    }
    \boiteQFQ{Question 2 :}{
      %question2
    }

\end{multicols}   
\vspace{-0.85cm}
\begin{multicols}{2}

    \boiteQFQ{Question 3 :}{
      %question3
    }
    \boiteQFA{
        \vspace{-0.35cm} 
        \begin{itemize}[itemsep=0.2em]
          \item[\raisebox{-0.1cm}{\begin{itembox} \textbf{1.} \end{itembox}}] %reponse1    
          \item[\raisebox{-0.1cm}{\begin{itembox} \textbf{2.} \end{itembox}}] %reponse2
          \item[\raisebox{-0.1cm}{\begin{itembox} \textbf{3.} \end{itembox}}] %reponse3
        \end{itemize}
    }
\end{multicols}   

\newpage 

%\begin{multicols}{2}
\boiteQFdet{Solution détaillée de la question 1 :}{

        %question1

        \tikz{\draw[dashed, line width=1pt] (0,0) -- (\linewidth,0);}

        \vspace{-0.25cm}\begin{multicols}{2}


        %details1

    \end{multicols}
}

\newpage

\boiteQFdet{Solution détaillée de la question 2 :}{
    
    %question2

    \tikz{\draw[dashed, line width=1pt] (0,0) -- (\linewidth,0);}


    \vspace{-0.25cm}\begin{multicols}{2}

        %details2

    \end{multicols}
}

%\columnbreak 
\newpage


\boiteQFdet{Solution détaillée de la question 3 :}{

    %question3

    \tikz{\draw[dashed, line width=1pt] (0,0) -- (\linewidth,0);}
    
    \vspace{-0.25cm}\begin{multicols}{2}

        %details3

    \end{multicols}
}
%\end{multicols}
\end{document}
