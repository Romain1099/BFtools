
\documentclass[12pt]{article}
\usepackage{geometry}

\geometry{paperwidth=18cm, paperheight=8.7cm, margin=0.2cm}

\usepackage{bfcours}
\usepackage{bfcours-fonts}
\usepackage{pagecolor} % Pour colorer le fond de la page
%\usetikzlibrary{backgrounds}
\usepackage{expl3} % Nécessaire pour utiliser le langage LaTeX3

% Retourne la valeur absolue d'un entier
\newcommand{\absnum}[1]{%
    \ifnum #1<0 \number\numexpr -#1\relax\else \number\numexpr #1\relax\fi%
}

% Retourne le minimum de deux entiers
\newcommand{\minnum}[2]{%
    \ifnum #1<#2 #1\else #2\fi%
}

% Calcul récursif du PGCD de deux entiers (les arguments doivent être entiers)
\newcommand{\gcdcalc}[2]{%
    \ifnum #2=0 %
    \absnum{#1}%
    \else %
    \gcdcalc{#2}{\the\numexpr #1 - (#1/#2)*#2 \relax}%
    \fi%
}
\ExplSyntaxOn
% Définition de la commande \mprefix
\cs_new:Npn \mprefix #1
{
    \int_case:nnF {#1}
    {
        {-12} { \text{pico} }
        {-9}  { \text{nano} }
        {-6}  { \text{micro} }
        {-3}  { \text{milli} }
        {-2}  { \text{centi} }
        {-1}  { \text{deci} }
        {0}   { \text{} } % Aucun préfixe pour la puissance 0
        {1}   { \text{deca} }
        {2}   { \text{hecto} }
        {3}   { \text{kilo} }
        {6}   { \text{mega} }
        {9}   { \text{giga} }
        {12}  { \text{tera} }
    }
    { \text{Valeur inconnue} } % Gestion des cas non pris en charge
}
\ExplSyntaxOff


% Calcul et affichage des diviseurs avec Lua
\begin{luacode*}
    function generate_divisors_table(n)
    local seen = {} -- Table pour suivre les nombres déjà rencontrés
    tex.print("\\noindent\\begin{tabular}{c|c}")
    for i = 1, n do
    if n % i == 0 then
    local complement = math.floor(n / i) -- Calcul de la partie entière
    -- Vérifier si le complément est déjà dans la table 'seen'
    if seen[complement] then break end
    -- Ajouter le diviseur et le complément dans la table
    seen[i] = true
    tex.print(i .. " & " .. complement .. " \\\\")
    end
    end
    tex.print("\\end{tabular}")
    end
\end{luacode*}

\newcommand{\divisorstable}[1]{%
    \directlua{generate_divisors_table(#1)}%
}

% Commande pour s'adapter en fonction du choix
\newcommand{\marchandcmd}[1]{%
    \IfStrEqCase{#1}{%
        {marchand}{\xdef\munite{articles}\def\intro{}\def\vend{vend}\def\tousSes{$\proportion$ de ses }\def\oeufs{$\quantite$\ articles}\def\labelvente{le nombre d'articles vendus}}
        {agriculteur}{\xdef\munite{tonnes}\def\intro{Suite à des intempéries,}\def\vend{perd}\def\tousSes{les $\proportion$èmes}\def\oeufs{sa récolte, estimée à \quantite\  tonnes}\def\labelvente{la masse perdue}}
        {boulanger}{\xdef\munite{baguettes}\def\intro{}\def\vend{vend}\def\tousSes{les $\proportion$èmes}\def\oeufs{de ses $\quantite$ baguettes}\def\labelvente{le nombre de baguettes vendues}}
    }[\textcolor{red}{\textbf{Métier inconnu : #1}}]% Si aucune correspondance
}

% Définition de la commande criteredivisibilite
\newcommand{\criteredivisibilite}[2]{%
    % #1 est le diviseur (n), #2 est le nombre à tester (N)
    \def\diviseur{#1}%
    \def\nombre{#2}%
    
    % Calcul du dernier chiffre du nombre
    \def\dernierchiffre{\intcalcMod{#2}{10}}
    
    % Calcul des deux derniers chiffres du nombre
    \def\deuxdernierchiffres{\intcalcMod{#2}{100}}
    
    % Vérifier quel diviseur a été spécifié
    \ifthenelse{\equal{\diviseur}{2}}{%
        % Cas 2
        \textbf{Critère de divisibilité par 2 :} Un nombre est divisible par 2 si son dernier chiffre est pair (0, 2, 4, 6 ou 8).
        
        \textbf{Application à \nombre :} Le dernier chiffre de \nombre{} est \dernierchiffre.
        
        \ifthenelse{\equal{\intcalcMod{\nombre}{2}}{0}}{%
            Comme le dernier chiffre est \dernierchiffre{} (qui est pair), \nombre{} est divisible par 2.
            }{%
            Comme le dernier chiffre est \dernierchiffre{} (qui n'est pas pair), \nombre{} n'est pas divisible par 2.
        }%
        }{%
        \ifthenelse{\equal{\diviseur}{3}}{%
            % Cas 3
            \textbf{Critère de divisibilité par 3 :} Un nombre est divisible par 3 si la somme de ses chiffres est divisible par 3.
            
            \textbf{Application à \nombre :}
            
            Pour vérifier si \nombre{} est divisible par 3, calculons la somme de ses chiffres.
            
            % Dans un cas réel, on calculerait la somme des chiffres
            \ifthenelse{\equal{\intcalcMod{\nombre}{3}}{0}}{%
                La somme des chiffres de \nombre{} est divisible par 3, donc \nombre{} est divisible par 3.
                }{%
                La somme des chiffres de \nombre{} n'est pas divisible par 3, donc \nombre{} n'est pas divisible par 3.
            }%
            }{%
            \ifthenelse{\equal{\diviseur}{4}}{%
                % Cas 4
                \textbf{Critère de divisibilité par 4 :} Un nombre est divisible par 4 si le nombre formé par ses deux derniers chiffres est divisible par 4.
                
                \textbf{Application à \nombre :}
                \ifthenelse{\nombre < 100}{%
                    Comme \nombre{} est inférieur à 100, nous vérifions directement si \nombre{} est divisible par 4.
                    
                    \ifthenelse{\equal{\intcalcMod{\nombre}{4}}{0}}{%
                        Le reste de la division de \nombre{} par 4 est 0, donc \nombre{} est divisible par 4.
                        }{%
                        Le reste de la division de \nombre{} par 4 est \intcalcMod{\nombre}{4}, donc \nombre{} n'est pas divisible par 4.
                    }%
                    }{%
                    Les deux derniers chiffres de \nombre{} forment le nombre \deuxdernierchiffres.
                    
                    \ifthenelse{\equal{\intcalcMod{\deuxdernierchiffres}{4}}{0}}{%
                        Comme \deuxdernierchiffres{} est divisible par 4, \nombre{} est divisible par 4.
                        }{%
                        Comme \deuxdernierchiffres{} n'est pas divisible par 4 (reste \intcalcMod{\deuxdernierchiffres}{4}), \nombre{} n'est pas divisible par 4.
                    }%
                }%
                }{%
                \ifthenelse{\equal{\diviseur}{5}}{%
                    % Cas 5
                    \textbf{Critère de divisibilité par 5 :} Un nombre est divisible par 5 si son dernier chiffre est 0 ou 5.
                    
                    \textbf{Application à \nombre :} Le dernier chiffre de \nombre{} est \dernierchiffre.
                    
                    \ifthenelse{\equal{\intcalcMod{\nombre}{5}}{0}}{%
                        Comme le dernier chiffre est \dernierchiffre{} (qui est
                        \ifthenelse{\equal{\dernierchiffre}{0}}{0}{5}), \nombre{} est divisible par 5.
                        }{%
                        Comme le dernier chiffre est \dernierchiffre, qui n'est ni 0 ni 5, \nombre{} n'est pas divisible par 5.
                    }%
                    }{%
                    \ifthenelse{\equal{\diviseur}{6}}{%
                        % Cas 6
                        \textbf{Critère de divisibilité par 6 :} Un nombre est divisible par 6 s'il est divisible à la fois par 2 et par 3.
                        
                        \textbf{Application à \nombre :}
                        
                        \textbf{Divisibilité par 2 :} Un nombre est divisible par 2 si son dernier chiffre est pair.
                        
                        \ifthenelse{\equal{\intcalcMod{\nombre}{2}}{0}}{%
                            Le dernier chiffre de \nombre{} est \dernierchiffre, qui est pair, donc \nombre{} est divisible par 2.
                            }{%
                            Le dernier chiffre de \nombre{} est \dernierchiffre, qui n'est pas pair, donc \nombre{} n'est pas divisible par 2.
                        }%
                        
                        \textbf{Divisibilité par 3 :} Un nombre est divisible par 3 si la somme de ses chiffres est divisible par 3.
                        
                        \ifthenelse{\equal{\intcalcMod{\nombre}{3}}{0}}{%
                            La somme des chiffres de \nombre{} est divisible par 3, donc \nombre{} est divisible par 3.
                            }{%
                            La somme des chiffres de \nombre{} n'est pas divisible par 3, donc \nombre{} n'est pas divisible par 3.
                        }%
                        
                        \textbf{Conclusion :}
                        
                        \ifthenelse{\equal{\intcalcMod{\nombre}{6}}{0}}{%
                            Comme \nombre{} est divisible à la fois par 2 et par 3, \nombre{} est divisible par 6.
                            }{%
                            \ifthenelse{\equal{\intcalcMod{\nombre}{2}}{0}}{%
                                Comme \nombre{} n'est pas divisible par 3, \nombre{} n'est pas divisible par 6.
                                }{%
                                Comme \nombre{} n'est pas divisible par 2, \nombre{} n'est pas divisible par 6.
                            }%
                        }%
                        }{%
                        \ifthenelse{\equal{\diviseur}{9}}{%
                            % Cas 9
                            \textbf{Critère de divisibilité par 9 :} Un nombre est divisible par 9 si la somme de ses chiffres est divisible par 9.
                            
                            \textbf{Application à \nombre :}
                            
                            Pour vérifier si \nombre{} est divisible par 9, calculons la somme de ses chiffres.
                            
                            % Pour les besoins de l'exemple, nous affichons directement le résultat
                            \ifthenelse{\equal{\intcalcMod{\nombre}{9}}{0}}{%
                                La somme des chiffres de \nombre{} est divisible par 9, donc \nombre{} est divisible par 9.
                                }{%
                                La somme des chiffres de \nombre{} n'est pas divisible par 9, donc \nombre{} n'est pas divisible par 9.
                            }%
                            }{%
                            % Diviseur invalide
                            %\textbf{Erreur :} Le diviseur \diviseur{} n'est pas dans la liste autorisée (2, 3, 4, 5, 6, 9).
                            Aucun critère ne fonctionne.
                        }%
                    }%
                }%
            }%
        }%
    }%
}


\ExplSyntaxOn

% Définition de la commande \para
% Si \na > 0, retourne \na ; si \na < 0, retourne (\na)
\cs_new:Npn \para #1
{
    \int_compare:nNnTF {#1} > {0}
    {#1} % Si #1 est positif, retourne #1
    {(#1)} % Si #1 est négatif, retourne (#1)
}

% Définition de la commande \expa
% Si \puisxa = 0, retourne \para x^{\puisxa}, sinon retourne \para
\cs_new:Npn \expa #1 #2
{
    \int_compare:nNnTF {#2} = {0}
    {\para{#1}} % Si la puissance est 0, affiche avec x^{#2}
    {
        \int_compare:nNnTF {#2} = {1}
        {\para{#1}~x}
        {\para{#1}~x^{#2}}
    } % Sinon, retourne simplement \para
}
\cs_new:Npn \expan #1 #2
{
    \int_compare:nNnTF {#2} = {0}
    {#1} % Si la puissance est 0, affiche avec x^{#2}
    {
        \int_compare:nNnTF {#2} = {1}
        {#1~x}
        {#1~x^{#2}}
    } % Sinon, retourne simplement \para
}
\cs_new:Npn \finalsigne #1
{
    \int_compare:nNnTF {#1} > {0}
    {+} % Si la puissance est 0, affiche avec x^{#2}
    {-} % Sinon, retourne simplement \para
}
\ExplSyntaxOff
% Définition de la couleur personnalisée
\definecolor{customBackground}{RGB}{0, 43, 54}

\def\boiteQFdetcouleur{customBackground}%{green!75!black}

\newcommand\boiteQFdet[2]{
    \begin{tcolorbox}[nobeforeafter,title=#1,halign title=flush left,fonttitle=\bfseries,colbacktitle=\boiteQFdetcouleur,colframe=customBackground,coltitle=white,colback=white,left=0.2pt,right=0.2pt,width=17.5cm]
        #2
    \end{tcolorbox}
}
\newcommand{\calculatriceAutorisee}{}
% On définit un booléen pour la détection
\newif\ifcalcAllowed
\newcommand\boiteQFQ[2]{%
    % Vérifie si #2 contient la chaîne "\calculatriceAutorisee"
    \IfSubStr{\detokenize{#2}}{\string\calculatriceAutorisee}{%
        % Si trouvé : titre avec l'icône
        \calcAllowedtrue
        }{%
        \calcAllowedfalse
    }%
    %\def\mytitle{#1}
    \begin{tcolorbox}[%
        nobeforeafter,
        title=#1,
        halign title=flush left,
        fonttitle=\bfseries,
        colbacktitle=customBackground,
        colframe=customBackground,
        coltitle=white,
        colback=white,
        left=0.2pt,
        right=0.2pt,
        height=4cm,
        width=8.2cm,
        % L'overlay ne modifie pas la géométrie du titre
        overlay={%
            \ifcalcAllowed
            % Place l'image en overlay à la droite du titre (ajustez xshift/yshift si besoin)
        \node[anchor=base west] at ([xshift=2.5cm,yshift=-0.325cm]title.west) {%xshift=-0.93cm,yshift=-0.325cm]title.east
            \includegraphics[width=0.8cm,height=0.8cm]{C:/Users/Utilisateur/Desktop/Faire/Macros/Programmes_de_facilitation/QF_generator/images/je_calcule.png}%
        };
        \fi
    }%
]
%\begin{itembox}\textbf{#1}\end{itembox}
#2
\end{tcolorbox}
}
%\def\boitereponsecouleur{gray!65!red}
\def\boitereponsecouleur{red!75!black}
\newcommand\boiteQFA[1]{
\begin{tcolorbox}[nobeforeafter,title=Réponses :,halign title=center,fonttitle=\bfseries,colbacktitle=\boitereponsecouleur,colframe=customBackground,coltitle=white,colback=white,left=5pt,right=0.2pt,height=4cm,width=8.2cm]
#1
\end{tcolorbox}
}

\def\myiconpath{}

\getRandomIconPath

\renewcommand\boiteQFA[1]{
\begin{tcolorbox}[nobeforeafter, title=Réponses :, halign title= flush left,
fonttitle=\bfseries, colbacktitle=\boitereponsecouleur,
coltitle=white, colback=white,colframe=customBackground, left=5pt, right=0.2pt,
height=4cm, width=8.2cm, enhanced, overlay={
    \node[anchor=center] at (frame.north west) [xshift=-0.5cm, yshift=0cm]
    {
        %\includegraphics[width=1.5cm]{../../../images/sticker_eclair.png}
        \includegraphics[width=1.5cm]{\myiconpath }
        %\randomincludegraphics[2cm][0cm][0cm]
    };
}]
#1
\end{tcolorbox}
}
\newcommand\boiteQzeroFA[1]{
\begin{tcolorbox}[nobeforeafter, title=Réponses :, halign title= flush left,
fonttitle=\bfseries, colbacktitle=\boitereponsecouleur,
coltitle=white, colback=white,colframe=customBackground, left=5pt, right=0.2pt,
height=4cm, width=8.2cm, enhanced, overlay={
\node[anchor=center] at (frame.north west) [xshift=-0.5cm, yshift=0cm]
{
    \includegraphics[width=1.5cm]{C:/Users/Utilisateur/Desktop/Faire/Macros/Programmes_de_facilitation/QF_generator/images/sticker_eclair.png}%{../../../images/sticker_eclair.png}
    %\includegraphics[width=1.5cm]{\myiconpath }
    %\randomincludegraphics[2cm][0cm][0cm]
};
}]
\begin{center}
\includegraphics[height=2.8cm]{\myiconpath }%
\end{center}
\end{tcolorbox}
}
\newtcolorbox{itembox}[1][]{
baseline,
nobeforeafter, % Supprime les espaces avant et après
colframe=customBackground,
colback=customBackground,
coltext=white,
coltitle=white,
boxsep=1pt,
boxrule=0.5pt,
arc=2pt,
left=2pt,
right=2pt,
top=1pt,
bottom=1pt,
width=0.7cm,
#1
}
\usepackage{graphicx}
\usepackage{fontawesome5}
\usepackage{siunitx} % Pour la notation avec virgule
\usepackage{xfp}
% Configurer siunitx pour utiliser la virgule comme séparateur décimal
%\sisetup{output-decimal-marker={,}}
\newcommand{\cperso}[1]{\num{\fpeval{ #1 }}}
% Configuration locale pour le formatage des nombres
\sisetup{%
%  round-mode=places,
%  round-precision=5,
output-decimal-marker={,},
%  group-separator={}
}

% Définir la commande \response
\newcommand{\fracrep}[2]{%
\FPeval{\result}{round(#1/#2:8)} % Calculer la fraction et arrondir à 4 décimales
\num{\result} % Afficher le résultat avec une virgule
}
\opset{decimalsepsymbol={,}}
%\newcommand{\degree}{^\circ}
\begin{document}
%niveau estimé : 2nde
%date de projection : 03_10_2025
%themes abordés : Racines carrées, Développer une expression (1), Pythagore
\pagecolor{customBackground}
\vspace{-0.5cm}
\begin{multicols}{2}
\boiteQFQ{Question 1 :}{
%version 12 - Court

\def\numa{49}
\def\numb{14}
\def\typeoperation{1}


\acc{Simplifier} l'écriture de la racine suivante :
\begin{center}
$\sqrt{\fpeval{\numa*\numb}}$
\end{center}
}
\boiteQFQ{Question 2 :}{
%version 21 - Court

\def\na{8}
\def\nb{-1}
\def\nc{-7}
\def\puisxa{2}
\def\puisxb{1}
\def\puisxc{0}
\def\signe{-}



%court
\def\expression{\expa{\na}{\puisxa} (\expa{\nb}{\puisxb} \signe \expa{\nc}{\puisxc})}
Développer l'expression suivante :
\begin{center}
$\expression$
\end{center}
}
\end{multicols}
\vspace{-0.85cm}
\begin{multicols}{2}

\boiteQFQ{Question 3 :}{
%version 10 - Court

\def\cota{25.84}
\def\diffhyp{21.42}



\edef\hyp{\fpeval{\cota+\diffhyp}}
\acc{Calculez la longueur} du côté manquant dans un \textbf{triangle rectangle} dont un côté de l'angle droit mesure $\affiche{\cota}$ cm et l'hypoténuse mesure $\affiche{\hyp}$ cm.

Donner l'arrondi au \acc{mm} près.
}
\boiteQzeroFA{}
\end{multicols}

\newpage
\vspace{-0.5cm}
\begin{multicols}{2}
\boiteQFQ{Question 1 :}{
%version 12 - Court

\def\numa{49}
\def\numb{14}
\def\typeoperation{1}


\acc{Simplifier} l'écriture de la racine suivante :
\begin{center}
$\sqrt{\fpeval{\numa*\numb}}$
\end{center}
}
\boiteQFQ{Question 2 :}{
%version 21 - Court

\def\na{8}
\def\nb{-1}
\def\nc{-7}
\def\puisxa{2}
\def\puisxb{1}
\def\puisxc{0}
\def\signe{-}



%court
\def\expression{\expa{\na}{\puisxa} (\expa{\nb}{\puisxb} \signe \expa{\nc}{\puisxc})}
Développer l'expression suivante :
\begin{center}
$\expression$
\end{center}
}

\end{multicols}
\vspace{-0.85cm}
\begin{multicols}{2}

\boiteQFQ{Question 3 :}{
%version 10 - Court

\def\cota{25.84}
\def\diffhyp{21.42}



\edef\hyp{\fpeval{\cota+\diffhyp}}
\acc{Calculez la longueur} du côté manquant dans un \textbf{triangle rectangle} dont un côté de l'angle droit mesure $\affiche{\cota}$ cm et l'hypoténuse mesure $\affiche{\hyp}$ cm.

Donner l'arrondi au \acc{mm} près.
}
\boiteQFA{
\vspace{-0.35cm}
\begin{itemize}[itemsep=0.2em]
\item[\raisebox{-0.1cm}{\begin{itembox} \textbf{1.} \end{itembox}}] \def\numa{49}\def\numb{14}\def\typeoperation{1}$\SimplificationRacine{\numa*\numb}$
\end{itemize}
}
\end{multicols}
\newpage
\vspace{-0.5cm}
\begin{multicols}{2}
\boiteQFQ{Question 1 :}{
%version 12 - Court

\def\numa{49}
\def\numb{14}
\def\typeoperation{1}


\acc{Simplifier} l'écriture de la racine suivante :
\begin{center}
$\sqrt{\fpeval{\numa*\numb}}$
\end{center}
}
\boiteQFQ{Question 2 :}{
%version 21 - Court

\def\na{8}
\def\nb{-1}
\def\nc{-7}
\def\puisxa{2}
\def\puisxb{1}
\def\puisxc{0}
\def\signe{-}



%court
\def\expression{\expa{\na}{\puisxa} (\expa{\nb}{\puisxb} \signe \expa{\nc}{\puisxc})}
Développer l'expression suivante :
\begin{center}
$\expression$
\end{center}
}

\end{multicols}
\vspace{-0.85cm}
\begin{multicols}{2}

\boiteQFQ{Question 3 :}{
%version 10 - Court

\def\cota{25.84}
\def\diffhyp{21.42}



\edef\hyp{\fpeval{\cota+\diffhyp}}
\acc{Calculez la longueur} du côté manquant dans un \textbf{triangle rectangle} dont un côté de l'angle droit mesure $\affiche{\cota}$ cm et l'hypoténuse mesure $\affiche{\hyp}$ cm.

Donner l'arrondi au \acc{mm} près.
}
\boiteQFA{
\vspace{-0.35cm}
\begin{itemize}[itemsep=0.2em]
\item[\raisebox{-0.1cm}{\begin{itembox} \textbf{1.} \end{itembox}}] \def\numa{49}\def\numb{14}\def\typeoperation{1}$\SimplificationRacine{\numa*\numb}$
\item[\raisebox{-0.1cm}{\begin{itembox} \textbf{2.} \end{itembox}}] \def\na{8}\def\nb{-1}\def\nc{-7}\def\puisxa{2}\def\puisxb{1}\def\puisxc{0}\def\signe{-}\def\numa{\num{\nombrebase} \times 10^{\puissance}}\def\numerique{\fpeval{\nombrebase*10^(\puissance)}}\def\numerreur{\fpeval{\nombrebase*10^(\decalage)}\times 10^{\fpeval{\puissance - \decalage}}}\def\indicator{\signe 1}\def\resexpr{    \expa{\fpeval{\para{\na}*\para{\nb}}}{\fpeval{\puisxa + \puisxb}} \finalsigne{(\indicator)*\fpeval{\para{\na}*\para{\nc}}} \expa{\fpeval{abs(\para{\na}*\para{\nc})}}{\fpeval{\puisxa + \puisxc}}}$\resexpr$
\end{itemize}
}
\end{multicols}
\newpage

\begin{multicols}{2}
\boiteQFQ{Question 1 :}{
%version 12 - Court

\def\numa{49}
\def\numb{14}
\def\typeoperation{1}


\acc{Simplifier} l'écriture de la racine suivante :
\begin{center}
$\sqrt{\fpeval{\numa*\numb}}$
\end{center}
}
\boiteQFQ{Question 2 :}{
%version 21 - Court

\def\na{8}
\def\nb{-1}
\def\nc{-7}
\def\puisxa{2}
\def\puisxb{1}
\def\puisxc{0}
\def\signe{-}



%court
\def\expression{\expa{\na}{\puisxa} (\expa{\nb}{\puisxb} \signe \expa{\nc}{\puisxc})}
Développer l'expression suivante :
\begin{center}
$\expression$
\end{center}
}

\end{multicols}
\vspace{-0.85cm}
\begin{multicols}{2}

\boiteQFQ{Question 3 :}{
%version 10 - Court

\def\cota{25.84}
\def\diffhyp{21.42}



\edef\hyp{\fpeval{\cota+\diffhyp}}
\acc{Calculez la longueur} du côté manquant dans un \textbf{triangle rectangle} dont un côté de l'angle droit mesure $\affiche{\cota}$ cm et l'hypoténuse mesure $\affiche{\hyp}$ cm.

Donner l'arrondi au \acc{mm} près.
}
\boiteQFA{
\vspace{-0.35cm}
\begin{itemize}[itemsep=0.2em]
\item[\raisebox{-0.1cm}{\begin{itembox} \textbf{1.} \end{itembox}}] \def\numa{49}\def\numb{14}\def\typeoperation{1}$\SimplificationRacine{\numa*\numb}$
\item[\raisebox{-0.1cm}{\begin{itembox} \textbf{2.} \end{itembox}}] \def\na{8}\def\nb{-1}\def\nc{-7}\def\puisxa{2}\def\puisxb{1}\def\puisxc{0}\def\signe{-}\def\numa{\num{\nombrebase} \times 10^{\puissance}}\def\numerique{\fpeval{\nombrebase*10^(\puissance)}}\def\numerreur{\fpeval{\nombrebase*10^(\decalage)}\times 10^{\fpeval{\puissance - \decalage}}}\def\indicator{\signe 1}\def\resexpr{    \expa{\fpeval{\para{\na}*\para{\nb}}}{\fpeval{\puisxa + \puisxb}} \finalsigne{(\indicator)*\fpeval{\para{\na}*\para{\nc}}} \expa{\fpeval{abs(\para{\na}*\para{\nc})}}{\fpeval{\puisxa + \puisxc}}}$\resexpr$
\item[\raisebox{-0.1cm}{\begin{itembox} \textbf{3.} \end{itembox}}] \def\cota{25.84}\def\diffhyp{21.42}\edef\hyp{\fpeval{\cota+\diffhyp}} $b \approx \affiche{\fpeval{round(sqrt(\hyp^2 - \cota^2),1)}}\ \mathrm{cm}$.
\end{itemize}
}
\end{multicols}

\newpage

%\begin{multicols}{2}
\boiteQFdet{Solution détaillée de la question 1 :}{

%version 12 - Court

\def\numa{49}
\def\numb{14}
\def\typeoperation{1}


\acc{Simplifier} l'écriture de la racine suivante :
\begin{center}
$\sqrt{\fpeval{\numa*\numb}}$
\end{center}

\tikz{\draw[dashed, line width=1pt] (0,0) -- (\linewidth,0);}

\vspace{-0.25cm}\setlength{\columnseprule}{0.4pt}\begin{multicols}{2}


%version 12 - solutions

\def\numa{49}
\def\numb{14}
\def\typeoperation{1}


Pour simplifier, on cherche les carrés parfaits dans la décomposition :
\begin{center}$\fpeval{\numa*\numb}=\Decomposition[Exposant]{\fpeval{\numa*\numb}}$\end{center}

Ensuite on utilise la formule : $\sqrt{ab} = \sqrt{a} \times \sqrt{b}$ (si $a,b \geq 0$) et on simplifie l'écriture des racines avec des termes au carré.

\columnbreak

Résultat simplifié : \begin{center}$\fpeval{\numa*\numb}=\SimplificationRacine{\numa*\numb}$\end{center}

\end{multicols}
}

\newpage

\boiteQFdet{Solution détaillée de la question 2 :}{

%version 21 - Court

\def\na{8}
\def\nb{-1}
\def\nc{-7}
\def\puisxa{2}
\def\puisxb{1}
\def\puisxc{0}
\def\signe{-}



%court
\def\expression{\expa{\na}{\puisxa} (\expa{\nb}{\puisxb} \signe \expa{\nc}{\puisxc})}
Développer l'expression suivante :
\begin{center}
$\expression$
\end{center}

\tikz{\draw[dashed, line width=1pt] (0,0) -- (\linewidth,0);}


\vspace{-0.25cm}\setlength{\columnseprule}{0.4pt}\begin{multicols}{2}

%version 21 - solutions

\def\na{8}
\def\nb{-1}
\def\nc{-7}
\def\puisxa{2}
\def\puisxb{1}
\def\puisxc{0}
\def\signe{-}



%help
\def\indicator{\signe 1}
\def\resexpr{
\expa{\fpeval{\para{\na}*\para{\nb}}}{\fpeval{\puisxa + \puisxb}} \finalsigne{(\indicator)*\fpeval{\para{\na}*\para{\nc}}} \expa{\fpeval{abs(\para{\na}*\para{\nc})}}{\fpeval{\puisxa + \puisxc}}
}
\def\colorexpression{{\color{red}\expa{\na}{\puisxa}} ({\color{blue}\expa{\nb}{\puisxb}} \signe {\color{purple}\expa{\nc}{\puisxc}})}

On utilise la formule de \acc{distributivité} :
\begin{center}
${\color{red}a} \times ({\color{blue}b} \signe {\color{purple}c}) = {\color{red}a}\times {\color{blue}b} \signe {\color{red}a}\times {\color{purple}c}$
\end{center}
avec : $\left\{ \begin{array}{l}
a = {\color{red}\expa{\na}{\puisxa}} \\
b = {\color{blue}\expa{\nb}{\puisxb}} \\
c = {\color{purple}\expa{\nc}{\puisxc}}
\end{array} \right.$

\columnbreak

Ainsi, l'expression \acc{développée} est : \\
\begin{align*}
&\colorexpression\\
= & {\color{red}\expa{\na}{\puisxa}}\times {\color{blue}\expa{\nb}{\puisxb}} \signe {\color{red}\expa{\na}{\puisxa}}\times {\color{purple}\expa{\nc}{\puisxc}}\\
= & \resexpr
\end{align*}

\end{multicols}
}

%\columnbreak
\newpage


\boiteQFdet{Solution détaillée de la question 3 :}{

%version 10 - Court

\def\cota{25.84}
\def\diffhyp{21.42}



\edef\hyp{\fpeval{\cota+\diffhyp}}
\acc{Calculez la longueur} du côté manquant dans un \textbf{triangle rectangle} dont un côté de l'angle droit mesure $\affiche{\cota}$ cm et l'hypoténuse mesure $\affiche{\hyp}$ cm.

Donner l'arrondi au \acc{mm} près.

\tikz{\draw[dashed, line width=1pt] (0,0) -- (\linewidth,0);}

\vspace{-0.25cm}\setlength{\columnseprule}{0.4pt}\begin{multicols}{2}

%version 10 - solutions

\def\cota{25.84}
\def\diffhyp{21.42}



Dans un \textbf{triangle rectangle}, l'\textbf{hypoténuse} $c$ est \textbf{calculée} à l'aide du \textbf{théorème de Pythagore}, qui énonce que	 $$c^2 = a^2 + b^2 $$ 	, où $a$ et $b$ sont les \textbf{côtés de l'angle droit}.

En remplaçant les longueurs connues :

$\num{\hyp}^2 = \affiche{\cota}^2 + b^2$

Ainsi la longueur cherchée est :

$b = \sqrt{\num{\hyp}^2 - \affiche{\cota}^2}$

\columnbreak


\textbf{Calculons} les carrés des côtés :

$ \affiche{\cota}^2 = \affiche{\fpeval{\cota^2}} $ \hfill et \hfill $ \affiche{\hyp}^2 = \affiche{\fpeval{\hyp^2}}$

\textbf{Soustrayons-les} :

$ \affiche{\hyp}^2 - \affiche{\cota}^2 = \affiche{\fpeval{\hyp^2 - \cota^2}}$

\textbf{En passant à la racine carrée}, on obtient la longueur manquante :

$b = \sqrt{\affiche{\fpeval{\hyp^2 - \cota^2}}} = \affiche{\fpeval{sqrt(\hyp^2 - \cota^2)}}$

Ainsi, $b \approx \affiche{\fpeval{round(sqrt(\hyp^2 - \cota^2),1)}}\ \mathrm{cm}$.

\end{multicols}
}
%\end{multicols}
\end{document}
