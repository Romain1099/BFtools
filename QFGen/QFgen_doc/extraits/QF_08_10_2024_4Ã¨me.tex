%niveau estimé : 4ème
    %date de projection : 08_10_2024
    %themes abordés : Addition, Évaluation, Équations
    
    \vspace{-0.5cm}
    \begin{multicols}{2}
        \boiteQFQ{Question 1 :}{
            \textbf{Calculer} la \textbf{somme} des nombres relatifs $-12$ et $-8$.
        }
        \boiteQFQ{Question 2 :}{
            \textbf{Calculer} la valeur de l'expression $E = 5x + 2$ lorsque $x = 3$.
        }
    \end{multicols}
    \vspace{-0.85cm}
    \begin{multicols}{2}
        
        \boiteQFQ{Question 3 :}{
            \textbf{Déterminer} le nombre \textbf{négatif} $x$ tel que $x^2 = 36$.
        }
        \boiteQFA{}
    \end{multicols}
    
    \newpage
    \vspace{-0.5cm}
    \begin{multicols}{2}
        \boiteQFQ{Question 1 :}{
            \textbf{Calculer} la \textbf{somme} des nombres relatifs $-12$ et $-8$.
        }
        \boiteQFQ{Question 2 :}{
            \textbf{Calculer} la valeur de l'expression $E = 5x + 2$ lorsque $x = 3$.
        }
        
    \end{multicols}
    \vspace{-0.85cm}
    \begin{multicols}{2}
        
        \boiteQFQ{Question 3 :}{
            \textbf{Déterminer} le nombre \textbf{négatif} $x$ tel que $x^2 = 36$.
        }
        \boiteQFA{
            \vspace{-0.35cm}
            \begin{itemize}[itemsep=0.2em]
                \item[\raisebox{-0.2cm}{\begin{itembox} \textbf{Q.1} \end{itembox}}] $-20$
            \end{itemize}
        }
    \end{multicols}
    \newpage
    \vspace{-0.5cm}
    \begin{multicols}{2}
        \boiteQFQ{Question 1 :}{
            \textbf{Calculer} la \textbf{somme} des nombres relatifs $-12$ et $-8$.
        }
        \boiteQFQ{Question 2 :}{
            \textbf{Calculer} la valeur de l'expression $E = 5x + 2$ lorsque $x = 3$.
        }
        
    \end{multicols}
    \vspace{-0.85cm}
    \begin{multicols}{2}
        
        \boiteQFQ{Question 3 :}{
            \textbf{Déterminer} le nombre \textbf{négatif} $x$ tel que $x^2 = 36$.
        }
        \boiteQFA{
            \vspace{-0.35cm}
            \begin{itemize}[itemsep=0.2em]
                \item[\raisebox{-0.2cm}{\begin{itembox} \textbf{Q.1} \end{itembox}}] $-20$
                \item[\raisebox{-0.2cm}{\begin{itembox} \textbf{Q.2} \end{itembox}}] $
                E = 5 \times 3 + 2 = 15 + 2 = 17
                $
            \end{itemize}
        }
    \end{multicols}
    \newpage
    
    \begin{multicols}{2}
        \boiteQFQ{Question 1 :}{
            \textbf{Calculer} la \textbf{somme} des nombres relatifs $-12$ et $-8$.
        }
        \boiteQFQ{Question 2 :}{
            \textbf{Calculer} la valeur de l'expression $E = 5x + 2$ lorsque $x = 3$.
        }
        
    \end{multicols}
    \vspace{-0.85cm}
    \begin{multicols}{2}
        
        \boiteQFQ{Question 3 :}{
            \textbf{Déterminer} le nombre \textbf{négatif} $x$ tel que $x^2 = 36$.
        }
        \boiteQFA{
            \vspace{-0.35cm}
            \begin{itemize}[itemsep=0.2em]
                \item[\raisebox{-0.2cm}{\begin{itembox} \textbf{Q.1} \end{itembox}}] $-20$
                \item[\raisebox{-0.2cm}{\begin{itembox} \textbf{Q.2} \end{itembox}}] $
                E = 5 \times 3 + 2 = 15 + 2 = 17
                $
                \item[\raisebox{-0.2cm}{\begin{itembox} \textbf{Q.3} \end{itembox}}] $x = -6$
            \end{itemize}
        }
    \end{multicols}
    
    \newpage
    
    %\begin{multicols}{2}
    \boiteQFdet{Question 1 :}{
        
        \textbf{Calculer} la \textbf{somme} des nombres relatifs $-12$ et $-8$.
        
        \tikz{\draw[dashed, line width=1pt] (0,0) -- (\linewidth,0);}
        
        \vspace{-0.25cm}\begin{multicols}{2}
        
        
        \textbf{Calculer} la \textbf{somme} des nombres relatifs $-12$ et $-8$ revient à \textbf{additionner} ces deux nombres :
        $$ (-12) + (-8) = -12 - 8 $$
        
        On applique la propriété de \textbf{somme} des nombres relatifs, selon laquelle additionner deux nombres négatifs revient à additionner leurs valeurs absolues avec le signe négatif :
        $$ \Rightarrow -12 - 8 = - (12 + 8) $$
        \[
            - (12 + 8) = -20
        \]
        
        Ainsi, la \textbf{somme} des nombres relatifs $-12$ et $-8$ est \textcolor{red}{$-20$}.
        
    \end{multicols}
}

\newpage

\boiteQFdet{Question 2 :}{
    
    \textbf{Calculer} la valeur de l'expression $E = 5x + 2$ lorsque $x = 3$.
    
    \tikz{\draw[dashed, line width=1pt] (0,0) -- (\linewidth,0);}
    
    
    \vspace{-0.25cm}\begin{multicols}{2}
    
    Pour \textbf{calculer} la valeur de l'expression $E = 5x + 2$ lorsque $x = 3$, on \textbf{remplace} $x$ par $3$ :
    
    \[
        E = 5 \times 3 + 2
    \]
    
    Effectuons les \textbf{calculs} :
    
    \[
        E = 15 + 2 = 17
    \]
    
    La valeur de l'expression est donc $\boxed{17}$.
    
\end{multicols}
}

%\columnbreak
\newpage

\boiteQFdet{Question 3 :}{

\textbf{Déterminer} le nombre \textbf{négatif} $x$ tel que $x^2 = 36$.

\tikz{\draw[dashed, line width=1pt] (0,0) -- (\linewidth,0);}

\vspace{-0.25cm}\begin{multicols}{2}

Pour \textbf{déterminer} le nombre négatif $x$ tel que $x^2 = 36$, on cherche déjà les nombres dont le carré vaut 36. \\

On remarque que $6^2 = 36$ et que $(-6)^2=36$\\


$6$ et $-6$ sont candidats, mais seul $-6$ est négatif.


Ainsi, la solution est $x= -6$.

\end{multicols}
}
%\end{multicols}