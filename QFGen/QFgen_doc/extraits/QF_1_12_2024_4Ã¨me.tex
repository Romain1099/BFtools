%niveau estimé : 4ème
%date de projection : 1_12_2024
%themes abordés : Développer une expression (1), Multiplier des fractions (1), Notation scientifique (1)
\pagecolor{customBackground}
\vspace{-0.5cm}
\begin{multicols}{2}
\boiteQFQ{Question 1 :}{
    %version 2 - Court
    
    \def\na{2}
    \def\nb{7}
    \def\nc{-8}
    \def\puisxa{2}
    \def\puisxb{1}
    \def\puisxc{0}
    \def\signe{+}
    
    
    
    %court
    \def\expression{\expa{\na}{\puisxa} (\expa{\nb}{\puisxb} \signe \expa{\nc}{\puisxc})}
    Développer l'expression suivante :
    \begin{center}
        $\expression$
    \end{center}
}
\boiteQFQ{Question 2 :}{
    %version 1 - Court
    
    \def\n{8}
    \def\d{10}
    \def\nb{4}
    \def\db{13}
    
    
    %court
    Effectuer le calcul suivant en donnant le résultat sous forme simplifiée :
    $$\dfrac{\n}{\d} + \dfrac{\nb}{\db}$$
}
\end{multicols}
\vspace{-0.85cm}
\begin{multicols}{2}

\boiteQFQ{Question 3 :}{
    %version 1 - Court
    
    \def\puissance{9}
    \def\nombrebase{5.312}
    \def\decalage{1}
    
    
    
    %court
    \def\numa{\num{\nombrebase} \times 10^{\puissance}}
    \def\numerique{\fpeval{\nombrebase*10^(\puissance)}}
    \def\numerreur{\num{\fpeval{\nombrebase*10^(\decalage)}}\times 10^{\fpeval{\puissance - \decalage}}}
    Calculer l'\acc{écriture décimale} du nombre suivant :
    \begin{center}
        $\numa$
    \end{center}
    Déterminer son \acc{ordre de grandeur} et le \acc{préfixe} qui lui est associé.
}
\boiteQFA{}
\end{multicols}

\newpage
\vspace{-0.5cm}
\begin{multicols}{2}
\boiteQFQ{Question 1 :}{
    %version 2 - Court
    
    \def\na{2}
    \def\nb{7}
    \def\nc{-8}
    \def\puisxa{2}
    \def\puisxb{1}
    \def\puisxc{0}
    \def\signe{+}
    
    
    
    %court
    \def\expression{\expa{\na}{\puisxa} (\expa{\nb}{\puisxb} \signe \expa{\nc}{\puisxc})}
    Développer l'expression suivante :
    \begin{center}
        $\expression$
    \end{center}
}
\boiteQFQ{Question 2 :}{
    %version 1 - Court
    
    \def\n{8}
    \def\d{10}
    \def\nb{4}
    \def\db{13}
    
    
    %court
    Effectuer le calcul suivant en donnant le résultat sous forme simplifiée :
    $$\dfrac{\n}{\d} + \dfrac{\nb}{\db}$$
}

\end{multicols}
\vspace{-0.85cm}
\begin{multicols}{2}

\boiteQFQ{Question 3 :}{
    %version 1 - Court
    
    \def\puissance{9}
    \def\nombrebase{5.312}
    \def\decalage{1}
    
    
    
    %court
    \def\numa{\num{\nombrebase} \times 10^{\puissance}}
    \def\numerique{\fpeval{\nombrebase*10^(\puissance)}}
    \def\numerreur{\num{\fpeval{\nombrebase*10^(\decalage)}}\times 10^{\fpeval{\puissance - \decalage}}}
    Calculer l'\acc{écriture décimale} du nombre suivant :
    \begin{center}
        $\numa$
    \end{center}
    Déterminer son \acc{ordre de grandeur} et le \acc{préfixe} qui lui est associé.
}
\boiteQFA{
    \vspace{-0.35cm}
    \begin{itemize}[itemsep=0.2em]
        \item[\raisebox{-0.1cm}{\begin{itembox} \textbf{1.} \end{itembox}}] \def\na{2}\def\nb{7}\def\nc{-8}\def\puisxa{2}\def\puisxb{1}\def\puisxc{0}\def\signe{+}\def\numa{\num{\nombrebase} \times 10^{\puissance}}\def\numerique{\fpeval{\nombrebase*10^(\puissance)}}\def\numerreur{\fpeval{\nombrebase*10^(\decalage)}\times 10^{\fpeval{\puissance - \decalage}}}\def\indicator{\signe 1}\def\resexpr{    \fpeval{\para{\na}*\para{\nb}}x^{\fpeval{\puisxa + \puisxb}} \finalsigne{(\indicator)*\fpeval{\para{\na}*\para{\nc}}} \fpeval{abs(\para{\na}*\para{\nc})}x^{\fpeval{\puisxa + \puisxc}}}$\resexpr$
    \end{itemize}
}
\end{multicols}
\newpage
\vspace{-0.5cm}
\begin{multicols}{2}
\boiteQFQ{Question 1 :}{
    %version 2 - Court
    
    \def\na{2}
    \def\nb{7}
    \def\nc{-8}
    \def\puisxa{2}
    \def\puisxb{1}
    \def\puisxc{0}
    \def\signe{+}
    
    
    
    %court
    \def\expression{\expa{\na}{\puisxa} (\expa{\nb}{\puisxb} \signe \expa{\nc}{\puisxc})}
    Développer l'expression suivante :
    \begin{center}
        $\expression$
    \end{center}
}
\boiteQFQ{Question 2 :}{
    %version 1 - Court
    
    \def\n{8}
    \def\d{10}
    \def\nb{4}
    \def\db{13}
    
    
    %court
    Effectuer le calcul suivant en donnant le résultat sous forme simplifiée :
    $$\dfrac{\n}{\d} + \dfrac{\nb}{\db}$$
}

\end{multicols}
\vspace{-0.85cm}
\begin{multicols}{2}

\boiteQFQ{Question 3 :}{
    %version 1 - Court
    
    \def\puissance{9}
    \def\nombrebase{5.312}
    \def\decalage{1}
    
    
    
    %court
    \def\numa{\num{\nombrebase} \times 10^{\puissance}}
    \def\numerique{\fpeval{\nombrebase*10^(\puissance)}}
    \def\numerreur{\num{\fpeval{\nombrebase*10^(\decalage)}}\times 10^{\fpeval{\puissance - \decalage}}}
    Calculer l'\acc{écriture décimale} du nombre suivant :
    \begin{center}
        $\numa$
    \end{center}
    Déterminer son \acc{ordre de grandeur} et le \acc{préfixe} qui lui est associé.
}
\boiteQFA{
    \vspace{-0.35cm}
    \begin{itemize}[itemsep=0.2em]
        \item[\raisebox{-0.1cm}{\begin{itembox} \textbf{1.} \end{itembox}}] \def\na{2}\def\nb{7}\def\nc{-8}\def\puisxa{2}\def\puisxb{1}\def\puisxc{0}\def\signe{+}\def\numa{\num{\nombrebase} \times 10^{\puissance}}\def\numerique{\fpeval{\nombrebase*10^(\puissance)}}\def\numerreur{\fpeval{\nombrebase*10^(\decalage)}\times 10^{\fpeval{\puissance - \decalage}}}\def\indicator{\signe 1}\def\resexpr{    \fpeval{\para{\na}*\para{\nb}}x^{\fpeval{\puisxa + \puisxb}} \finalsigne{(\indicator)*\fpeval{\para{\na}*\para{\nc}}} \fpeval{abs(\para{\na}*\para{\nc})}x^{\fpeval{\puisxa + \puisxc}}}$\resexpr$
        \item[\raisebox{-0.1cm}{\begin{itembox} \textbf{2.} \end{itembox}}] \def\n{10}    \def\d{10}    \def\nb{4}    \def\db{13}        \def\nresa{\fpeval{\n + \nb}}    \def\nresb{\fpeval{\n * \db + \nb * \d}}    \def\nresc{\fpeval{\n + \nb * \d / \db}}    \def\nresd{\fpeval{\n * \db + \nb * \d}}    \def\dresa{\fpeval{\d + \db}}    \def\dresb{\fpeval{\d * \db}}    \def\dresc{\fpeval{\d}}    \def\dresd{\fpeval{\d + \db}}    $\Simplification{\nresb}{\dresb}$
    \end{itemize}
}
\end{multicols}
\newpage

\begin{multicols}{2}
\boiteQFQ{Question 1 :}{
    %version 2 - Court
    
    \def\na{2}
    \def\nb{7}
    \def\nc{-8}
    \def\puisxa{2}
    \def\puisxb{1}
    \def\puisxc{0}
    \def\signe{+}
    
    
    
    %court
    \def\expression{\expa{\na}{\puisxa} (\expa{\nb}{\puisxb} \signe \expa{\nc}{\puisxc})}
    Développer l'expression suivante :
    \begin{center}
        $\expression$
    \end{center}
}
\boiteQFQ{Question 2 :}{
    %version 1 - Court
    
    \def\n{8}
    \def\d{10}
    \def\nb{4}
    \def\db{13}
    
    
    %court
    Effectuer le calcul suivant en donnant le résultat sous forme simplifiée :
    $$\dfrac{\n}{\d} + \dfrac{\nb}{\db}$$
}

\end{multicols}
\vspace{-0.85cm}
\begin{multicols}{2}

\boiteQFQ{Question 3 :}{
    %version 1 - Court
    
    \def\puissance{9}
    \def\nombrebase{5.312}
    \def\decalage{1}
    
    
    
    %court
    \def\numa{\num{\nombrebase} \times 10^{\puissance}}
    \def\numerique{\fpeval{\nombrebase*10^(\puissance)}}
    \def\numerreur{\num{\fpeval{\nombrebase*10^(\decalage)}}\times 10^{\fpeval{\puissance - \decalage}}}
    Calculer l'\acc{écriture décimale} du nombre suivant :
    \begin{center}
        $\numa$
    \end{center}
    Déterminer son \acc{ordre de grandeur} et le \acc{préfixe} qui lui est associé.
}
\boiteQFA{
    \vspace{-0.35cm}
    \begin{itemize}[itemsep=0.2em]
        \item[\raisebox{-0.1cm}{\begin{itembox} \textbf{1.} \end{itembox}}] \def\na{2}\def\nb{7}\def\nc{-8}\def\puisxa{2}\def\puisxb{1}\def\puisxc{0}\def\signe{+}\def\numa{\num{\nombrebase} \times 10^{\puissance}}\def\numerique{\fpeval{\nombrebase*10^(\puissance)}}\def\numerreur{\fpeval{\nombrebase*10^(\decalage)}\times 10^{\fpeval{\puissance - \decalage}}}\def\indicator{\signe 1}\def\resexpr{    \fpeval{\para{\na}*\para{\nb}}x^{\fpeval{\puisxa + \puisxb}} \finalsigne{(\indicator)*\fpeval{\para{\na}*\para{\nc}}} \fpeval{abs(\para{\na}*\para{\nc})}x^{\fpeval{\puisxa + \puisxc}}}$\resexpr$
        \item[\raisebox{-0.1cm}{\begin{itembox} \textbf{2.} \end{itembox}}] \def\n{10}    \def\d{10}    \def\nb{4}    \def\db{13}        \def\nresa{\fpeval{\n + \nb}}    \def\nresb{\fpeval{\n * \db + \nb * \d}}    \def\nresc{\fpeval{\n + \nb * \d / \db}}    \def\nresd{\fpeval{\n * \db + \nb * \d}}    \def\dresa{\fpeval{\d + \db}}    \def\dresb{\fpeval{\d * \db}}    \def\dresc{\fpeval{\d}}    \def\dresd{\fpeval{\d + \db}}    $\Simplification{\nresb}{\dresb}$
        \item[\raisebox{-0.1cm}{\begin{itembox} \textbf{3.} \end{itembox}}] \def\puissance{9}\def\nombrebase{5.312}\def\decalage{1}\def\numa{\num{\nombrebase} \times 10^{\puissance}}\def\numerique{\fpeval{\nombrebase*10^(\puissance)}}\def\numerreur{\fpeval{\nombrebase*10^(\decalage)}\times 10^{\fpeval{\puissance - \decalage}}}$\num{\numerique}$ \hfill - $10^{\puissance}$ - \hfill \mprefix{\puissance}
    \end{itemize}
}
\end{multicols}

\newpage

%\begin{multicols}{2}
\boiteQFdet{Solution détaillée de la question 1 :}{

%version 2 - Court

\def\na{2}
\def\nb{7}
\def\nc{-8}
\def\puisxa{2}
\def\puisxb{1}
\def\puisxc{0}
\def\signe{+}



%court
\def\expression{\expa{\na}{\puisxa} (\expa{\nb}{\puisxb} \signe \expa{\nc}{\puisxc})}
Développer l'expression suivante :
\begin{center}
    $\expression$
\end{center}

\tikz{\draw[dashed, line width=1pt] (0,0) -- (\linewidth,0);}

\vspace{-0.25cm}\begin{multicols}{2}


%version 2 - solutions

\def\na{2}
\def\nb{7}
\def\nc{-8}
\def\puisxa{2}
\def\puisxb{1}
\def\puisxc{0}
\def\signe{+}



%help
\def\indicator{\signe 1}
\def\resexpr{
    \expa{\fpeval{\para{\na}*\para{\nb}}}{\fpeval{\puisxa + \puisxb}} \finalsigne{(\indicator)*\fpeval{\para{\na}*\para{\nc}}} \expa{\fpeval{abs(\para{\na}*\para{\nc})}}{\fpeval{\puisxa + \puisxc}}
}
\def\colorexpression{{\color{red}\expa{\na}{\puisxa}} ({\color{blue}\expa{\nb}{\puisxb}} \signe {\color{purple}\expa{\nc}{\puisxc}})}

On utilise la formule de \acc{distributivité} :
\begin{center}
    ${\color{red}a} \times ({\color{blue}b} \signe {\color{purple}c}) = {\color{red}a}\times {\color{blue}b} \signe {\color{red}a}\times {\color{purple}c}$
\end{center}
avec : $\left\{ \begin{array}{l}
    a = {\color{red}\expa{\na}{\puisxa}} \\
    b = {\color{blue}\expa{\nb}{\puisxb}} \\
    c = {\color{purple}\expa{\nc}{\puisxc}}
\end{array} \right.$

\columnbreak

Ainsi, l'expression \acc{développée} est : \\
\begin{align*}
    &\colorexpression\\
    = & {\color{red}\expa{\na}{\puisxa}}\times {\color{blue}\expa{\nb}{\puisxb}} \signe {\color{red}\expa{\na}{\puisxa}}\times {\color{purple}\expa{\nc}{\puisxc}}\\
    = & \resexpr
\end{align*}

\end{multicols}
}

\newpage

\boiteQFdet{Solution détaillée de la question 2 :}{

%version 1 - Court

\def\n{8}
\def\d{10}
\def\nb{4}
\def\db{13}


%court
Effectuer le calcul suivant en donnant le résultat sous forme simplifiée :
$$\dfrac{\n}{\d} + \dfrac{\nb}{\db}$$

\tikz{\draw[dashed, line width=1pt] (0,0) -- (\linewidth,0);}


\vspace{-0.25cm}\begin{multicols}{2}

%version 1 - solutions

\def\n{8}
\def\d{10}
\def\nb{4}
\def\db{13}


%help

% Définitions des macros
\def\decnum{\Decomposition[Longue]{\fpeval{\n * \db + \nb * \d}}}
\def\decden{\Decomposition[Longue]{\fpeval{\d * \db}}}

\begin{align*}
\dfrac{\n}{\d} + \dfrac{\nb}{\db}
&=~ \dfrac{\n \times \db}{\d \times \db} + \dfrac{\nb \times \d}{\db \times \d}\\
&=~ \dfrac{\fpeval{\n * \db + \nb * \d}}{\fpeval{\d * \db}}\\
&=~ \dfrac{\decnum}{\decden}\\
&=~ \Simplification{\fpeval{\n * \db + \nb * \d}}{\fpeval{\d * \db}}
\end{align*}

\end{multicols}
}

%\columnbreak
\newpage


\boiteQFdet{Solution détaillée de la question 3 :}{

%version 1 - Court

\def\puissance{9}
\def\nombrebase{5.312}
\def\decalage{1}



%court
\def\numa{\num{\nombrebase} \times 10^{\puissance}}
\def\numerique{\fpeval{\nombrebase*10^(\puissance)}}
\def\numerreur{\num{\fpeval{\nombrebase*10^(\decalage)}}\times 10^{\fpeval{\puissance - \decalage}}}
Calculer l'\acc{écriture décimale} du nombre suivant :
\begin{center}
$\numa$
\end{center}
Déterminer son \acc{ordre de grandeur} et le \acc{préfixe} qui lui est associé.

\tikz{\draw[dashed, line width=1pt] (0,0) -- (\linewidth,0);}

\vspace{-0.25cm}\begin{multicols}{2}

%version 1 - solutions

\def\puissance{9}
\def\nombrebase{5.312}
\def\decalage{1}



%help
\def\numa{\num{\nombrebase} \times 10^{\puissance}}
\def\numerique{\fpeval{\nombrebase*10^(\puissance)}}
\def\numerreur{\num{\fpeval{\nombrebase*10^(\decalage)}}\times 10^{\fpeval{\puissance - \decalage}}}

Le nombre $\numa$ correspond au nombre $\num{\numerique}$ en \acc{écriture décimale}.\\
Ce nombre est l'\acc{ordre de grandeur} $10^{\puissance}$ qui correspond au
\acc{préfixe} \mprefix{\puissance}

\end{multicols}
}
%\end{multicols}