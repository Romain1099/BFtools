%niveau estimé : 4ème
    %date de projection : 10_10_2024
    %themes abordés : Algèbre, Équations, Addition
    
    \vspace{-0.5cm}
    \begin{multicols}{2}
        \boiteQFQ{Question 1 :}{
            \textbf{Calculer} la valeur de l'expression $3x - 7$ pour $x = -5$.
        }
        \boiteQFQ{Question 2 :}{
            \textbf{Trouver} le nombre positif $x$ tel que $x^2=64$.
        }
    \end{multicols}
    \vspace{-0.85cm}
    \begin{multicols}{2}
        
        \boiteQFQ{Question 3 :}{
            \textbf{Calculer} la somme des fractions $\dfrac{3}{4}$ et $\dfrac{5}{12}$.
        }
        \boiteQFA{}
    \end{multicols}
    
    \newpage
    \vspace{-0.5cm}
    \begin{multicols}{2}
        \boiteQFQ{Question 1 :}{
            \textbf{Calculer} la valeur de l'expression $3x - 7$ pour $x = -5$.
        }
        \boiteQFQ{Question 2 :}{
            \textbf{Trouver} le nombre positif $x$ tel que $x^2=64$.
        }
        
    \end{multicols}
    \vspace{-0.85cm}
    \begin{multicols}{2}
        
        \boiteQFQ{Question 3 :}{
            \textbf{Calculer} la somme des fractions $\dfrac{3}{4}$ et $\dfrac{5}{12}$.
        }
        \boiteQFA{
            \vspace{-0.35cm}
            \begin{itemize}[itemsep=0.2em]
                \item[\raisebox{-0.2cm}{\begin{itembox} \textbf{Q.1} \end{itembox}}] $\boxed{-22}$
            \end{itemize}
        }
    \end{multicols}
    \newpage
    \vspace{-0.5cm}
    \begin{multicols}{2}
        \boiteQFQ{Question 1 :}{
            \textbf{Calculer} la valeur de l'expression $3x - 7$ pour $x = -5$.
        }
        \boiteQFQ{Question 2 :}{
            \textbf{Trouver} le nombre positif $x$ tel que $x^2=64$.
        }
        
    \end{multicols}
    \vspace{-0.85cm}
    \begin{multicols}{2}
        
        \boiteQFQ{Question 3 :}{
            \textbf{Calculer} la somme des fractions $\dfrac{3}{4}$ et $\dfrac{5}{12}$.
        }
        \boiteQFA{
            \vspace{-0.35cm}
            \begin{itemize}[itemsep=0.2em]
                \item[\raisebox{-0.2cm}{\begin{itembox} \textbf{Q.1} \end{itembox}}] $\boxed{-22}$
                \item[\raisebox{-0.2cm}{\begin{itembox} \textbf{Q.2} \end{itembox}}] $\mathbf{x = 8} $
            \end{itemize}
        }
    \end{multicols}
    \newpage
    
    \begin{multicols}{2}
        \boiteQFQ{Question 1 :}{
            \textbf{Calculer} la valeur de l'expression $3x - 7$ pour $x = -5$.
        }
        \boiteQFQ{Question 2 :}{
            \textbf{Trouver} le nombre positif $x$ tel que $x^2=64$.
        }
        
    \end{multicols}
    \vspace{-0.85cm}
    \begin{multicols}{2}
        
        \boiteQFQ{Question 3 :}{
            \textbf{Calculer} la somme des fractions $\dfrac{3}{4}$ et $\dfrac{5}{12}$.
        }
        \boiteQFA{
            \vspace{-0.35cm}
            \begin{itemize}[itemsep=0.2em]
                \item[\raisebox{-0.2cm}{\begin{itembox} \textbf{Q.1} \end{itembox}}] $\boxed{-22}$
                \item[\raisebox{-0.2cm}{\begin{itembox} \textbf{Q.2} \end{itembox}}] $\mathbf{x = 8} $
                \item[\raisebox{-0.2cm}{\begin{itembox} \textbf{Q.3} \end{itembox}}] $\dfrac{7}{6}$
            \end{itemize}
        }
    \end{multicols}
    
    \newpage
    
    %\begin{multicols}{2}
    \boiteQFdet{Question 1 :}{
        
        \textbf{Calculer} la valeur de l'expression $3x - 7$ pour $x = -5$.
        
        \tikz{\draw[dashed, line width=1pt] (0,0) -- (\linewidth,0);}
        
        \vspace{-0.25cm}\begin{multicols}{2}
        
        
        \textbf{Calculons} la valeur de l'expression $3x - 7$ pour $x = -5$.
        
        Remplaçons $x$ par $-5$ :
        
        $$3(-5) - 7.$$
        
        \textbf{Traitons} les opérations :
        
        - \textbf{Multiplions} : $3 \times (-5) = -15$.
        
        - \textbf{Soustrayons} : $-15 - 7 = -22$.
        
        Ainsi, la valeur de l'expression pour $x = -5$ est $\boxed{-22}$.
        
    \end{multicols}
}

\newpage

\boiteQFdet{Question 2 :}{
    
    \textbf{Trouver} le nombre positif $x$ tel que $x^2=64$.
    
    \tikz{\draw[dashed, line width=1pt] (0,0) -- (\linewidth,0);}
    
    
    \vspace{-0.25cm}\begin{multicols}{2}
    
    Pour \textbf{trouver} le nombre positif $x$ tel que $x^2 = 64$, nous devons \textbf{résoudre} l'équation:
    
    \[
        x^2 = 64
    \]
    
    En \textbf{appliquant} la racine carrée des deux côtés, nous obtenons:
    
    \[
        x = \sqrt{64}
    \]
    
    Puisque nous cherchons le nombre positif,
    
    \[
        x = 8
    \]
    
    Ainsi, le nombre positif est $\mathbf{x = 8}$.
    
\end{multicols}
}

%\columnbreak
\newpage

\boiteQFdet{Question 3 :}{

\textbf{Calculer} la somme des fractions $\dfrac{3}{4}$ et $\dfrac{5}{12}$.

\tikz{\draw[dashed, line width=1pt] (0,0) -- (\linewidth,0);}

\vspace{-0.25cm}\begin{multicols}{2}

\textbf{Pour additionner} deux fractions, comme $\dfrac{3}{4}$ et $\dfrac{5}{12}$, nous devons \textbf{utiliser} un dénominateur commun.

Le \textbf{plus petit commun multiple} des dénominateurs $4$ et $12$ est $12$.

\textbf{Transformons} la première fraction :
\[
    \dfrac{3}{4} = \dfrac{3 \times 3}{4 \times 3} = \dfrac{9}{12}
\]

Ainsi, la somme est :
\[
    \dfrac{9}{12} + \dfrac{5}{12} = \dfrac{9+5}{12}
\]

Enfin, \textbf{calculons} la somme :
\[
    \dfrac{14}{12}
\]

\textbf{Simplifions} la fraction $\dfrac{14}{12}$ :
\[
    \dfrac{14}{12} = \dfrac{14 \div 2}{12 \div 2} = \dfrac{7}{6}
\]

\textbf{La somme des fractions} est donc $\dfrac{7}{6}$.

\end{multicols}
}
%\end{multicols}