\documentclass[a4paper,11pt,fleqn]{article}

\usepackage[left=1cm,right=0.5cm,top=0.5cm,bottom=2cm]{geometry}

\usepackage{bfcours}
\usepackage{bfcours-fonts}
%\usepackage{bfcours-fonts-dys}

\def\rdifficulty{1}
\setrdexo{%left skip=1cm,
display exotitle,
exo header = tcolorbox,
%display tags,
skin = bouyachakka,
lower ={box=crep},
display score,
display level,
save lower,
score=\points,
level=\rdifficulty,
overlay={\node[inner sep=0pt,
anchor=west,rotate=90, yshift=0.3cm]%,xshift=-3em], yshift=0.45cm
at (frame.south west) {\thetags[0]} ;}
]%obligatoire
}
\setrdcrep{seyes, correction=true, correction color=monrose, correction font = \large\bfseries}

\newcommand{\tikzinclude}[1]{%
    \stepcounter{tikzfigcounter}%
    \csname tikzfig#1\endcsname
}
\input{enonce_figures}

\hypersetup{
    pdfauthor={R.Deschamps},
    pdfsubject={},
    pdfkeywords={},
    pdfproducer={LuaLaTeX},
    pdfcreator={Boum Factory}
}

\begin{document}

\setcounter{pagecounter}{0}
\setcounter{ExoMA}{0}
\setcounter{prof}{0}

\def\points{\phantom{AAA}}
\def\difficulty{\phantom{AAA}}
\chapitre[
    % niveau : $\mathbf{6^{\text{ème}}}$,$\mathbf{5^{\text{ème}}}$,$\mathbf{4^{\text{ème}}}$,$\mathbf{3^{\text{ème}}}$,$\mathbf{2^{\text{nde}}}$,$\mathbf{1^{\text{ère}}}$,$\mathbf{T^{\text{Le}}}$,
    ]{
    % Nom_sequence : ,Equations
    }{
    % type_etablissement : Collège,Lycée,
    }{
    % nom_etablissement : Amadis Jamyn,Eugène Belgrand,CBPM
    }{
    % supplement : ,\tableauPresenteEvalSixieme{}{10},\tableofcontents
    }{
    Plan de séquence :% type_document_principal : Exercices
    }

    \begin{tikzpicture}[]
        \SequenceItem{Introduction}{%Rappels de calculs littéral :
            \begin{itemize}[label=$\bullet$]
                \item Questions Flash
            \end{itemize}
        }{Devoirs : \begin{itemize}[label=$\bullet$]
            \item 
        \end{itemize}}
        \SequenceItem[below = 1.7cm of desc1.south west]{Cours\\\large{$+$}\\Exercices}{
            \begin{itemize}[label=$\bullet$,itemsep=0em]
                \item Questions Flash
            \end{itemize}
        }{
            Devoirs : \begin{itemize}[label=$\bullet$]
                \item 
            \end{itemize}
        }
        \SequenceItem[below = 1.7cm of desc2.south west]{Cours\\\large{$+$}\\Exercices}{%Nature d'une égalité
        \begin{itemize}[label=$\bullet$,itemsep=0em]
            \item Questions Flash
            \item 
        \end{itemize}
        }{
            Devoirs : \begin{itemize}[label=$\bullet$]
                \item 
            \end{itemize}
        }
        \SequenceItem[below = 1.7cm of desc3.south west]{Activité\\\large{$+$}\\\'Eval \no 1}{%Les bouteilles :\\
        Introduction à la distributivité\\

        \textbf{Test de connaissances \no 1 :} \\Début du cours
        }{
        }
        \SequenceItem[below = 1.7cm of desc4.south west]{Cours\\\large{$+$}\\Exercices}{%Distributivité
        \begin{itemize}[label=$\bullet$,itemsep=0em]
            \item 
        \end{itemize}
        }{
            Devoirs : \begin{itemize}[label=$\bullet$]
                \item 
            \end{itemize}
        }
        \SequenceItem[below = 1.7cm of desc5.south west]{Cours\\\large{$+$}\\Exercices}{%Réduire et factoriser
        \begin{itemize}[label=$\bullet$,itemsep=0em]
            \item 
        \end{itemize}
        }{
            Devoirs : \begin{itemize}[label=$\bullet$]
                \item 
            \end{itemize}
        }
    \SequenceItem[below = 1.7cm of desc6.south west]{Cours\\\large{$+$}\\\'Eval n°2}{%Distributivité double
        \begin{itemize}[label=$\bullet$,itemsep=0em]
            \item 
        \end{itemize}
        }{
            Devoirs : \begin{itemize}[label=$\bullet$]
                \item 
            \end{itemize}
        }
    \SequenceItem[below = 1.7cm of desc7.south west]{\'Evaluation\\\textbf{ou}\\Exercices}{
        \begin{itemize}[label=$\bullet$,itemsep=0em]
            \item \textbf{\'Evaluation \no 3 :} \\Ensemble du chapitre
        \end{itemize}
        }{
            %Selon l'avancée de la classe, il est possible de rajouter une séance d'exercices.
        }
    \end{tikzpicture}

    \tableaurecapsequence{
        \recapHeures%

        \enteteContenu%

        \competence{Connaître le vocabulaire lié au calcul littéral}
    }

    

% contenu_principal : %\section{Extraits de productions}
\newcounter{qcounter} % Déclare un nouveau compteur
\setcounter{qcounter}{0}
\newcommand{\myy}{%
  \stepcounter{qcounter}% Incrémente le compteur
  \theqcounter% Affiche la valeur actuelle du compteur
}

%\input{extraits/QF_18_11_2024_4ème.tex}

%\input{extraits/QF_18_10_2024_4ème.tex}

%\newpage
%\pagecolor{customBackground}
%\section*{Série \myy}


%\newpage
%\pagecolor{customBackground}

%\input{extraits/QF_19_11_2024_4ème.tex}

%\newpage
\pagecolor{customBackground}
\section*{Série \myy}


\newpage
\pagecolor{customBackground}

\input{extraits/QF_25_11_2024_4ème.tex}

\newpage
\pagecolor{customBackground}
\section*{Série \myy}


\newpage
\pagecolor{customBackground}

\input{extraits/QF_26_11_2024_4ème.tex}

\newpage
\pagecolor{customBackground}
\section*{Série \myy}


\newpage
\pagecolor{customBackground}

\input{extraits/QF_28_11_2024_4ème.tex}

\newpage
\pagecolor{customBackground}
\section*{Série \myy}


\newpage
\pagecolor{customBackground}

\input{extraits/QF_02_12_2024_4ème.tex}

\newpage
\pagecolor{customBackground}
\section*{Série \myy}


\newpage
\pagecolor{customBackground}

\input{extraits/QF_1_12_2024_4ème.tex}

,\inputImprim{enonce},


\end{document}