%niveau estimé : 4ème
    %date de projection : 11_10_2024
    %themes abordés : substitution, Arithmétique., pourcentage
    
    \vspace{-0.5cm}
    \begin{multicols}{2}
        \boiteQFQ{Question 1 :}{
            \textbf{Remplacer} $x$ par $-3$ dans l'expression suivante et \textbf{calculer} la valeur : $5x^2 - 4x + 7$.
        }
        \boiteQFQ{Question 2 :}{
            \textbf{Calculez} la valeur de l'expression : $-(5{-}3) + (7 - 9) - (-2) + 4$.
        }
    \end{multicols}
    \vspace{-0.85cm}
    \begin{multicols}{2}
        
        \boiteQFQ{Question 3 :}{
            \textbf{Calculer} le \textbf{nouveau prix} d'un article coûtant $30\,\text{\euro{}}$ après une \textbf{réduction} de $15\%$.
        }
        \boiteQFA{}
    \end{multicols}
    
    \newpage
    \vspace{-0.5cm}
    \begin{multicols}{2}
        \boiteQFQ{Question 1 :}{
            \textbf{Remplacer} $x$ par $-3$ dans l'expression suivante et \textbf{calculer} la valeur : $5x^2 - 4x + 7$.
        }
        \boiteQFQ{Question 2 :}{
            \textbf{Calculez} la valeur de l'expression : $-(5{-}3) + (7 - 9) - (-2) + 4$.
        }
        
    \end{multicols}
    \vspace{-0.85cm}
    \begin{multicols}{2}
        
        \boiteQFQ{Question 3 :}{
            \textbf{Calculer} le \textbf{nouveau prix} d'un article coûtant $30\,\text{\euro{}}$ après une \textbf{réduction} de $15\%$.
        }
        \boiteQFA{
            \vspace{-0.35cm}
            \begin{itemize}[itemsep=0.2em]
                \item[\raisebox{-0.2cm}{\begin{itembox} \textbf{Q.1} \end{itembox}}] $5\times(-3)^2 - 4\times(-3) + 7 = 64$
            \end{itemize}
        }
    \end{multicols}
    \newpage
    \vspace{-0.5cm}
    \begin{multicols}{2}
        \boiteQFQ{Question 1 :}{
            \textbf{Remplacer} $x$ par $-3$ dans l'expression suivante et \textbf{calculer} la valeur : $5x^2 - 4x + 7$.
        }
        \boiteQFQ{Question 2 :}{
            \textbf{Calculez} la valeur de l'expression : $-(5{-}3) + (7 - 9) - (-2) + 4$.
        }
        
    \end{multicols}
    \vspace{-0.85cm}
    \begin{multicols}{2}
        
        \boiteQFQ{Question 3 :}{
            \textbf{Calculer} le \textbf{nouveau prix} d'un article coûtant $30\,\text{\euro{}}$ après une \textbf{réduction} de $15\%$.
        }
        \boiteQFA{
            \vspace{-0.35cm}
            \begin{itemize}[itemsep=0.2em]
                \item[\raisebox{-0.2cm}{\begin{itembox} \textbf{Q.1} \end{itembox}}] $5\times(-3)^2 - 4\times(-3) + 7 = 64$
                \item[\raisebox{-0.2cm}{\begin{itembox} \textbf{Q.2} \end{itembox}}] $2$
            \end{itemize}
        }
    \end{multicols}
    \newpage
    
    \begin{multicols}{2}
        \boiteQFQ{Question 1 :}{
            \textbf{Remplacer} $x$ par $-3$ dans l'expression suivante et \textbf{calculer} la valeur : $5x^2 - 4x + 7$.
        }
        \boiteQFQ{Question 2 :}{
            \textbf{Calculez} la valeur de l'expression : $-(5{-}3) + (7 - 9) - (-2) + 4$.
        }
        
    \end{multicols}
    \vspace{-0.85cm}
    \begin{multicols}{2}
        
        \boiteQFQ{Question 3 :}{
            \textbf{Calculer} le \textbf{nouveau prix} d'un article coûtant $30\,\text{\euro{}}$ après une \textbf{réduction} de $15\%$.
        }
        \boiteQFA{
            \vspace{-0.35cm}
            \begin{itemize}[itemsep=0.2em]
                \item[\raisebox{-0.2cm}{\begin{itembox} \textbf{Q.1} \end{itembox}}] $5\times(-3)^2 - 4\times(-3) + 7 = 64$
                \item[\raisebox{-0.2cm}{\begin{itembox} \textbf{Q.2} \end{itembox}}] $2$
                \item[\raisebox{-0.2cm}{\begin{itembox} \textbf{Q.3} \end{itembox}}] $\textbf{Nouveau prix} = 25{,}5 \, \text{\euro{}} $
            \end{itemize}
        }
    \end{multicols}
    
    \newpage
    
    %\begin{multicols}{2}
    \boiteQFdet{Question 1 :}{
        
        \textbf{Remplacer} $x$ par $-3$ dans l'expression suivante et \textbf{calculer} la valeur : $5x^2 - 4x + 7$.
        
        \tikz{\draw[dashed, line width=1pt] (0,0) -- (\linewidth,0);}
        
        \vspace{-0.25cm}\begin{multicols}{2}
        
        
        Pour \textbf{calculer} la valeur de l'expression $5x^2 - 4x + 7$ lorsque $x = -3$, on procède comme suit :
        
        1. \textbf{Remplacer} $x$ par $-3$ dans l'expression :
        \[
            5\times(-3)^2 - 4\times(-3) + 7
        \]
        
        2. \textbf{Calculer} chaque terme :
        \[
            5 \times (-3)^2 = 5 \times 9 = 45
        \]
        \[
            -4 \times (-3) = 12
        \]
        
        3. \textbf{Additionner} les termes :
        \[
            45 + 12 + 7 = \fpeval{45 + 12 + 7}
        \]
        
        Ainsi, la valeur de l'expression est $\fpeval{45 + 12 + 7}$.
        
    \end{multicols}
}

\newpage

\boiteQFdet{Question 2 :}{
    
    \textbf{Calculez} la valeur de l'expression : $-(5{-}3) + (7 - 9) - (-2) + 4$.
    
    \tikz{\draw[dashed, line width=1pt] (0,0) -- (\linewidth,0);}
    
    
    \vspace{-0.25cm}\begin{multicols}{2}
    
    L'expression à \textbf{calculer} est :\\
    $ -(5{-}3) + (7 - 9) - (-2) + 4 $\\
    \textbf{Évaluons} chaque partie séparément : \\
    $-(5-3) = -2$ \\
    $(7 - 9) = -2$ \\
    $-(-2) = 2$ \\
    \textbf{Ajoutons} les valeurs obtenues et $4$ : \\
    $-2 + (-2) + 2 + 4 = \mathbf{2}$
    
    
    \text{La valeur de l'expression est donc \textbf{2}.}
    
\end{multicols}
}

%\columnbreak
\newpage

\boiteQFdet{Question 3 :}{

\textbf{Calculer} le \textbf{nouveau prix} d'un article coûtant $30\,\text{\euro{}}$ après une \textbf{réduction} de $15\%$.

\tikz{\draw[dashed, line width=1pt] (0,0) -- (\linewidth,0);}

\vspace{-0.25cm}\begin{multicols}{2}

Le \textbf{prix initial} de l'article est $30\,\text{\euro{}}$. \\
La \textbf{réduction} appliquée est de $15\%$, ce qui correspond à \\
$30\times\dfrac {15}{100} = 4{,}5 \text{\euro{}}$.

Ainsi, le \textbf{nouveau prix}, après \textbf{réduction}, est :

\[
    \text{{Nouveau prix}} = 30 - 4{,}5 =  25{,}5  \, \text{\euro{}}
\]

Le \textbf{nouveau prix} est donc $\fpeval{30 * 0.85} \, \text{\euro{}}$.

\end{multicols}
}
%\end{multicols}