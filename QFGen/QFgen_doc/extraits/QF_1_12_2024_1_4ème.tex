%niveau estimé : 4ème
%date de projection : 1_12_2024
%themes abordés : Déterminer l'inverse d'une fraction, Déterminer un pourcentage, Pourcentage augmentation réduction
\pagecolor{customBackground}
\vspace{-0.5cm}
\begin{multicols}{2}
\boiteQFQ{Question 1 :}{
    %version 1 - Court
    
    \def\na{7}
    \def\da{4}
    
    
    
    
    %court
    
    \acc{Déterminer} l'inverse de la fraction $\dfrac{\na}{\da}$
}
\boiteQFQ{Question 2 :}{
    %version 1 - Court
    
    \def\he{5}
    \def\mi{50}
    \def\dia{111}
    
    
    %court
    Dans une série de $\he$ h $\mi$ min, il y a $\dia$ min de dialogues.
    Quel est le pourcentage de dialogues ?\\
}
\end{multicols}
\vspace{-0.85cm}
\begin{multicols}{2}

\boiteQFQ{Question 3 :}{
    %version 1 - Court
    
    \def\na{82.8}
    \def\augmentation{-1}
    
    
    
    %court
    \def\checkaugmentation{
        \ifthenelse{\augmentation > 0}
        { % Cas augmentation
            \def\operation{augmentation}
            \def\operationtext{augmenté}
            \def\calculation{\fpeval{\na * (1 + abs(\augmentation) / 100)}}
            \def\rounded{\fpeval{round(\na * (1 + abs(\augmentation) / 100), 2)}}
            
            \def\sign{+}
        }
        { % Cas réduction
            \def\operation{réduction}
            \def\operationtext{diminué}
            \def\calculation{\fpeval{\na * (1 - abs(\augmentation) / 100)}}
            
            \def\rounded{\fpeval{round(\na * (1 - abs(\augmentation) / 100 ), 2)}}
            \def\sign{-}
        }
    }
    \checkaugmentation
    Déterminer le nouveau prix d'un article coutant $\num{\na}~$\euro{} et dont le nouveau prix a une \operation\  de $\fpeval{abs(\augmentation)}~\%$
}
\boiteQFA{}
\end{multicols}

\newpage
\vspace{-0.5cm}
\begin{multicols}{2}
\boiteQFQ{Question 1 :}{
    %version 1 - Court
    
    \def\na{7}
    \def\da{4}
    
    
    
    
    %court
    
    \acc{Déterminer} l'inverse de la fraction $\dfrac{\na}{\da}$
}
\boiteQFQ{Question 2 :}{
    %version 1 - Court
    
    \def\he{5}
    \def\mi{50}
    \def\dia{111}
    
    
    %court
    Dans une série de $\he$ h $\mi$ min, il y a $\dia$ min de dialogues.
    Quel est le pourcentage de dialogues ?\\
}

\end{multicols}
\vspace{-0.85cm}
\begin{multicols}{2}

\boiteQFQ{Question 3 :}{
    %version 1 - Court
    
    \def\na{82.8}
    \def\augmentation{-1}
    
    
    
    %court
    \def\checkaugmentation{
        \ifthenelse{\augmentation > 0}
        { % Cas augmentation
            \def\operation{augmentation}
            \def\operationtext{augmenté}
            \def\calculation{\fpeval{\na * (1 + abs(\augmentation) / 100)}}
            \def\rounded{\fpeval{round(\na * (1 + abs(\augmentation) / 100), 2)}}
            
            \def\sign{+}
        }
        { % Cas réduction
            \def\operation{réduction}
            \def\operationtext{diminué}
            \def\calculation{\fpeval{\na * (1 - abs(\augmentation) / 100)}}
            
            \def\rounded{\fpeval{round(\na * (1 - abs(\augmentation) / 100 ), 2)}}
            \def\sign{-}
        }
    }
    \checkaugmentation
    Déterminer le nouveau prix d'un article coutant $\num{\na}~$\euro{} et dont le nouveau prix a une \operation\  de $\fpeval{abs(\augmentation)}~\%$
}
\boiteQFA{
    \vspace{-0.35cm}
    \begin{itemize}[itemsep=0.2em]
        \item[\raisebox{-0.1cm}{\begin{itembox} \textbf{1.} \end{itembox}}] \def\na{7}    \def\da{4}            $\dfrac{\da}{\na}$
    \end{itemize}
}
\end{multicols}
\newpage
\vspace{-0.5cm}
\begin{multicols}{2}
\boiteQFQ{Question 1 :}{
    %version 1 - Court
    
    \def\na{7}
    \def\da{4}
    
    
    
    
    %court
    
    \acc{Déterminer} l'inverse de la fraction $\dfrac{\na}{\da}$
}
\boiteQFQ{Question 2 :}{
    %version 1 - Court
    
    \def\he{5}
    \def\mi{50}
    \def\dia{111}
    
    
    %court
    Dans une série de $\he$ h $\mi$ min, il y a $\dia$ min de dialogues.
    Quel est le pourcentage de dialogues ?\\
}

\end{multicols}
\vspace{-0.85cm}
\begin{multicols}{2}

\boiteQFQ{Question 3 :}{
    %version 1 - Court
    
    \def\na{82.8}
    \def\augmentation{-1}
    
    
    
    %court
    \def\checkaugmentation{
        \ifthenelse{\augmentation > 0}
        { % Cas augmentation
            \def\operation{augmentation}
            \def\operationtext{augmenté}
            \def\calculation{\fpeval{\na * (1 + abs(\augmentation) / 100)}}
            \def\rounded{\fpeval{round(\na * (1 + abs(\augmentation) / 100), 2)}}
            
            \def\sign{+}
        }
        { % Cas réduction
            \def\operation{réduction}
            \def\operationtext{diminué}
            \def\calculation{\fpeval{\na * (1 - abs(\augmentation) / 100)}}
            
            \def\rounded{\fpeval{round(\na * (1 - abs(\augmentation) / 100 ), 2)}}
            \def\sign{-}
        }
    }
    \checkaugmentation
    Déterminer le nouveau prix d'un article coutant $\num{\na}~$\euro{} et dont le nouveau prix a une \operation\  de $\fpeval{abs(\augmentation)}~\%$
}
\boiteQFA{
    \vspace{-0.35cm}
    \begin{itemize}[itemsep=0.2em]
        \item[\raisebox{-0.1cm}{\begin{itembox} \textbf{1.} \end{itembox}}] \def\na{7}    \def\da{4}            $\dfrac{\da}{\na}$
        \item[\raisebox{-0.1cm}{\begin{itembox} \textbf{2.} \end{itembox}}] \def\he{5}    \def\mi{50}    \def\dia{111}        $\num{\fpeval{round(100 * \dia / (\he * 60 + \mi),1)}}~\%$
    \end{itemize}
}
\end{multicols}
\newpage

\begin{multicols}{2}
\boiteQFQ{Question 1 :}{
    %version 1 - Court
    
    \def\na{7}
    \def\da{4}
    
    
    
    
    %court
    
    \acc{Déterminer} l'inverse de la fraction $\dfrac{\na}{\da}$
}
\boiteQFQ{Question 2 :}{
    %version 1 - Court
    
    \def\he{5}
    \def\mi{50}
    \def\dia{111}
    
    
    %court
    Dans une série de $\he$ h $\mi$ min, il y a $\dia$ min de dialogues.
    Quel est le pourcentage de dialogues ?\\
}

\end{multicols}
\vspace{-0.85cm}
\begin{multicols}{2}

\boiteQFQ{Question 3 :}{
    %version 1 - Court
    
    \def\na{82.8}
    \def\augmentation{-1}
    
    
    
    %court
    \def\checkaugmentation{
        \ifthenelse{\augmentation > 0}
        { % Cas augmentation
            \def\operation{augmentation}
            \def\operationtext{augmenté}
            \def\calculation{\fpeval{\na * (1 + abs(\augmentation) / 100)}}
            \def\rounded{\fpeval{round(\na * (1 + abs(\augmentation) / 100), 2)}}
            
            \def\sign{+}
        }
        { % Cas réduction
            \def\operation{réduction}
            \def\operationtext{diminué}
            \def\calculation{\fpeval{\na * (1 - abs(\augmentation) / 100)}}
            
            \def\rounded{\fpeval{round(\na * (1 - abs(\augmentation) / 100 ), 2)}}
            \def\sign{-}
        }
    }
    \checkaugmentation
    Déterminer le nouveau prix d'un article coutant $\num{\na}~$\euro{} et dont le nouveau prix a une \operation\  de $\fpeval{abs(\augmentation)}~\%$
}
\boiteQFA{
    \vspace{-0.35cm}
    \begin{itemize}[itemsep=0.2em]
        \item[\raisebox{-0.1cm}{\begin{itembox} \textbf{1.} \end{itembox}}] \def\na{7}    \def\da{4}            $\dfrac{\da}{\na}$
        \item[\raisebox{-0.1cm}{\begin{itembox} \textbf{2.} \end{itembox}}] \def\he{5}    \def\mi{50}    \def\dia{111}        $\num{\fpeval{round(100 * \dia / (\he * 60 + \mi),1)}}~\%$
        \item[\raisebox{-0.1cm}{\begin{itembox} \textbf{3.} \end{itembox}}] \def\na{82.8}    \def\augmentation{-1}            \def\checkaugmentation{        \ifthenelse{\augmentation > 0}        {             \def\operation{augmentation}            \def\operationtext{augmenté}            \def\calculation{\fpeval{\na * (1 + abs(\augmentation) / 100)}}            \def\rounded{\fpeval{round(\na * (1 + abs(\augmentation) / 100), 2)}}            \def\sign{+}        }        {           \def\operation{réduction}            \def\operationtext{diminué}            \def\calculation{\fpeval{\na * (1 - abs(\augmentation) / 100)}}            \def\rounded{\fpeval{round(\na * (1 - abs(\augmentation) / 100), 2)}}            \def\sign{-}        }    }    \checkaugmentation    $\num{\rounded}$
    \end{itemize}
}
\end{multicols}

\newpage

%\begin{multicols}{2}
\boiteQFdet{Solution détaillée de la question 1 :}{

%version 1 - Court

\def\na{7}
\def\da{4}




%court

\acc{Déterminer} l'inverse de la fraction $\dfrac{\na}{\da}$

\tikz{\draw[dashed, line width=1pt] (0,0) -- (\linewidth,0);}

\vspace{-0.25cm}\begin{multicols}{2}


%version 1 - solutions

\def\na{7}
\def\da{4}




%help

L'inverse de $\dfrac{\na}{\da}$ est $\dfrac{\da}{\na}$

\end{multicols}
}

\newpage

\boiteQFdet{Solution détaillée de la question 2 :}{

%version 1 - Court

\def\he{5}
\def\mi{50}
\def\dia{111}


%court
Dans une série de $\he$ h $\mi$ min, il y a $\dia$ min de dialogues.
Quel est le pourcentage de dialogues ?\\

\tikz{\draw[dashed, line width=1pt] (0,0) -- (\linewidth,0);}


\vspace{-0.25cm}\begin{multicols}{2}

%version 1 - solutions

\def\he{5}
\def\mi{50}
\def\dia{111}


%help

Dans la série, il y a $\num{\fpeval{\he * 60 + \mi}}~$minutes de dialogues.


\begin{tcbtab}[Tableau de proportionnalité]{p{2.5cm}|p{1.5cm}|p{1.5cm}}
\cellcolor{gray!10}\bfseries Quantité & $\dia$ & \color{red} \bfseries \num{\fpeval{round(100 * \dia / (\he * 60 + \mi),1)}}\\\hline
\cellcolor{gray!10}\bfseries Total& \num{\fpeval{\he * 60 + \mi}} & 100\\
\end{tcbtab}

Ainsi, $\num{\fpeval{\he * 60 + \mi}}~$minutes de dialogues représentent $\num{\fpeval{round(100 * \dia / (\he * 60 + \mi),1)}}~\%$ de la série.


%Ou encore : \\
%$\dfrac{\dia}{\minutefilm} = \Simplification{\dia}{\fpeval{\he * 60 + \mi}} = \dfrac{\fpeval{round(100 * \dia / \he * 60 + \mi ,1)}}{100}$

\end{multicols}
}

%\columnbreak
\newpage


\boiteQFdet{Solution détaillée de la question 3 :}{

%version 1 - Court

\def\na{82.8}
\def\augmentation{-1}



%court
\def\checkaugmentation{
\ifthenelse{\augmentation > 0}
{ % Cas augmentation
\def\operation{augmentation}
\def\operationtext{augmenté}
\def\calculation{\fpeval{\na * (1 + abs(\augmentation) / 100)}}
\def\rounded{\fpeval{round(\na * (1 + abs(\augmentation) / 100), 2)}}

\def\sign{+}
}
{ % Cas réduction
\def\operation{réduction}
\def\operationtext{diminué}
\def\calculation{\fpeval{\na * (1 - abs(\augmentation) / 100)}}

\def\rounded{\fpeval{round(\na * (1 - abs(\augmentation) / 100 ), 2)}}
\def\sign{-}
}
}
\checkaugmentation
Déterminer le nouveau prix d'un article coutant $\num{\na}~$\euro{} et dont le nouveau prix a une \operation\  de $\fpeval{abs(\augmentation)}~\%$

\tikz{\draw[dashed, line width=1pt] (0,0) -- (\linewidth,0);}

\vspace{-0.25cm}\begin{multicols}{2}

%version 1 - solutions

\def\na{82.8}
\def\augmentation{-1}



%help

% Définitions des macros
% Définir le texte et le calcul en fonction du signe
\def\checkaugmentation{
\ifthenelse{\augmentation > 0}
{ % Cas augmentation
\def\operation{augmentation}
\def\operationtext{augmenté}
\def\calculation{\fpeval{\na * (1 + abs(\augmentation) / 100)}}
\def\rounded{\fpeval{round(\na * (1 + abs(\augmentation) / 100), 2)}}

\def\sign{+}
}
{ % Cas réduction
\def\operation{réduction}
\def\operationtext{diminué}
\def\calculation{\fpeval{\na * (1 - abs(\augmentation) / 100)}}

\def\rounded{\fpeval{round(\na * (1 - abs(\augmentation) / 100), 2)}}
\def\sign{-}
}
}

\def\afficherresultat{
\ifthenelse{\equal{\calculation}{\rounded}}
{Nouveau Prix $= \num{\rounded}$\\} % Si égaux
{Nouveau Prix $= \num{\calculation} \approx \num{\rounded}$\\
On donne l'arrondi au \textbf{centième} près car il s'agit d'un prix en euros.} % Si différents
}
\checkaugmentation
Pour \textbf{calculer} le \textbf{nouveau prix} après une \operation\  de $\fpeval{abs(\augmentation)}~\%$, on utilise la formule suivante : \\

Nouveau Prix $= \text{Ancien Prix} \sign \text{Ancien Prix} \times \left( \dfrac{t}{100}\right)  $ \\

où $t$ est le taux d'augmentation ou de réduction.

Ici il s'agit d'une \operation, avec $\text{Ancien Prix} = \num{\na}$ et $t = \num{\fpeval{abs(\augmentation)}}$.\\

Nouveau Prix $= \num{\na} \sign \num{\na} \times \dfrac{\num{\fpeval{abs(\augmentation)}}}{100}$\\

Nouveau Prix $=\num{\na} \sign \num{\fpeval{\na * abs(\augmentation) / 100}} $\\

Ainsi  :\\
\afficherresultat

\end{multicols}
}
%\end{multicols}