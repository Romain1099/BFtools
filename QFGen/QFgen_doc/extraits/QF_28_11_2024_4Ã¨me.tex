%niveau estimé : 4ème
%date de projection : 28_11_2024
%themes abordés : , ,
\pagecolor{customBackground}
\vspace{-0.5cm}
\begin{multicols}{2}
\boiteQFQ{Question 1 :}{
    \def\n{8}
    \def\d{6}
    \def\nb{5}
    \def\db{9}
    
    
    %court
    Effectuer le calcul suivant en donnant le résultat sous forme simplifiée :
    $$\dfrac{\n}{\d} + \dfrac{\nb}{\db}$$
}
\boiteQFQ{Question 2 :}{
    \def\na{101.65}
    \def\augmentation{-49}
    \def\checkaugmentation{
        \ifthenelse{\augmentation > 0}
        { % Cas augmentation
            \def\operation{augmentation}
            \def\operationtext{augmenté}
            \def\calculation{\fpeval{\na * (1 + abs(\augmentation) / 100)}}
            \def\rounded{\fpeval{round(\na * (1 + abs(\augmentation) / 100), 2)}}
            
            \def\sign{+}
        }
        { % Cas réduction
            \def\operation{réduction}
            \def\operationtext{diminué}
            \def\calculation{\fpeval{\na * (1 - abs(\augmentation) / 100)}}
            
            \def\rounded{\fpeval{round(\na * (1 - abs(\augmentation) / 100 ), 2)}}
            \def\sign{-}
        }
    }
    \checkaugmentation
    Déterminer le nouveau prix d'un article coutant $\num{\na}~$\euro{} et dont le nouveau prix a une \operation\  de $\fpeval{abs(\augmentation)}~\%$
}
\end{multicols}
\vspace{-0.85cm}
\begin{multicols}{2}

\boiteQFQ{Question 3 :}{
    \def\he{3}
    \def\mi{11}
    \def\dia{61}
    
    
    %court
    Dans une série de $\he$ h $\mi$ min, il y a $\dia$ min de dialogues.\\
    Quel est le pourcentage de dialogues ?
}
\boiteQFA{}
\end{multicols}

\newpage
\vspace{-0.5cm}
\begin{multicols}{2}
\boiteQFQ{Question 1 :}{
    \def\n{8}
    \def\d{6}
    \def\nb{5}
    \def\db{9}
    
    
    %court
    Effectuer le calcul suivant en donnant le résultat sous forme simplifiée :
    $$\dfrac{\n}{\d} + \dfrac{\nb}{\db}$$
}
\boiteQFQ{Question 2 :}{
    \def\na{101.65}
    \def\augmentation{-49}
    \def\checkaugmentation{
        \ifthenelse{\augmentation > 0}
        { % Cas augmentation
            \def\operation{augmentation}
            \def\operationtext{augmenté}
            \def\calculation{\fpeval{\na * (1 + abs(\augmentation) / 100)}}
            \def\rounded{\fpeval{round(\na * (1 + abs(\augmentation) / 100), 2)}}
            
            \def\sign{+}
        }
        { % Cas réduction
            \def\operation{réduction}
            \def\operationtext{diminué}
            \def\calculation{\fpeval{\na * (1 - abs(\augmentation) / 100)}}
            
            \def\rounded{\fpeval{round(\na * (1 - abs(\augmentation) / 100 ), 2)}}
            \def\sign{-}
        }
    }
    \checkaugmentation
    Déterminer le nouveau prix d'un article coutant $\num{\na}~$\euro{} et dont le nouveau prix a une \operation\  de $\fpeval{abs(\augmentation)}~\%$
}

\end{multicols}
\vspace{-0.85cm}
\begin{multicols}{2}

\boiteQFQ{Question 3 :}{
    \def\he{3}
    \def\mi{11}
    \def\dia{61}
    
    
    %court
    Dans une série de $\he$ h $\mi$ min, il y a $\dia$ min de dialogues.\\
    Quel est le pourcentage de dialogues ?
}
\boiteQFA{
    \vspace{-0.35cm}
    \begin{itemize}[itemsep=0.2em]
        \item[\raisebox{-0.1cm}{\begin{itembox} \textbf{1.} \end{itembox}}] \def\n{8}
        \def\d{6}
        \def\nb{5}
        \def\db{9}
        $\Simplification{\fpeval{\n * \db + \nb * \d}}{\fpeval{\d * \db}}$
    \end{itemize}
}
\end{multicols}
\newpage
\vspace{-0.5cm}
\begin{multicols}{2}
\boiteQFQ{Question 1 :}{
    \def\n{8}
    \def\d{6}
    \def\nb{5}
    \def\db{9}
    
    
    %court
    Effectuer le calcul suivant en donnant le résultat sous forme simplifiée :
    $$\dfrac{\n}{\d} + \dfrac{\nb}{\db}$$
}
\boiteQFQ{Question 2 :}{
    \def\na{101.65}
    \def\augmentation{-49}
    \def\checkaugmentation{
        \ifthenelse{\augmentation > 0}
        { % Cas augmentation
            \def\operation{augmentation}
            \def\operationtext{augmenté}
            \def\calculation{\fpeval{\na * (1 + abs(\augmentation) / 100)}}
            \def\rounded{\fpeval{round(\na * (1 + abs(\augmentation) / 100), 2)}}
            
            \def\sign{+}
        }
        { % Cas réduction
            \def\operation{réduction}
            \def\operationtext{diminué}
            \def\calculation{\fpeval{\na * (1 - abs(\augmentation) / 100)}}
            
            \def\rounded{\fpeval{round(\na * (1 - abs(\augmentation) / 100 ), 2)}}
            \def\sign{-}
        }
    }
    \checkaugmentation
    Déterminer le nouveau prix d'un article coutant $\num{\na}~$\euro{} et dont le nouveau prix a une \operation\  de $\fpeval{abs(\augmentation)}~\%$
}

\end{multicols}
\vspace{-0.85cm}
\begin{multicols}{2}

\boiteQFQ{Question 3 :}{
    \def\he{3}
    \def\mi{11}
    \def\dia{61}
    
    
    %court
    Dans une série de $\he$ h $\mi$ min, il y a $\dia$ min de dialogues.\\
    Quel est le pourcentage de dialogues ?
}
\boiteQFA{
    \vspace{-0.35cm}
    \begin{itemize}[itemsep=0.2em]
        \item[\raisebox{-0.1cm}{\begin{itembox} \textbf{1.} \end{itembox}}] \def\n{8}
        \def\d{6}
        \def\nb{5}
        \def\db{9}
        $\Simplification{\fpeval{\n * \db + \nb * \d}}{\fpeval{\d * \db}}$
        \item[\raisebox{-0.1cm}{\begin{itembox} \textbf{2.} \end{itembox}}] \def\na{101.65}
        \def\augmentation{-49}
        \def\checkaugmentation{
            \ifthenelse{\augmentation > 0}
            { % Cas augmentation
                \def\operation{augmentation}
                \def\operationtext{augmenté}
                \def\calculation{\fpeval{\na * (1 + abs(\augmentation) / 100)}}
                \def\rounded{\fpeval{round(\na * (1 + abs(\augmentation) / 100), 2)}}
                
                \def\sign{+}
            }
            { % Cas réduction
                \def\operation{réduction}
                \def\operationtext{diminué}
                \def\calculation{\fpeval{\na * (1 - abs(\augmentation) / 100)}}
                
                \def\rounded{\fpeval{round(\na * (1 - abs(\augmentation) / 100 ), 2)}}
                \def\sign{-}
            }
        }
        \checkaugmentation
        \num{\rounded}
    \end{itemize}
}
\end{multicols}
\newpage

\begin{multicols}{2}
\boiteQFQ{Question 1 :}{
    \def\n{8}
    \def\d{6}
    \def\nb{5}
    \def\db{9}
    
    
    %court
    Effectuer le calcul suivant en donnant le résultat sous forme simplifiée :
    $$\dfrac{\n}{\d} + \dfrac{\nb}{\db}$$
}
\boiteQFQ{Question 2 :}{
    \def\na{101.65}
    \def\augmentation{-49}
    \def\checkaugmentation{
        \ifthenelse{\augmentation > 0}
        { % Cas augmentation
            \def\operation{augmentation}
            \def\operationtext{augmenté}
            \def\calculation{\fpeval{\na * (1 + abs(\augmentation) / 100)}}
            \def\rounded{\fpeval{round(\na * (1 + abs(\augmentation) / 100), 2)}}
            
            \def\sign{+}
        }
        { % Cas réduction
            \def\operation{réduction}
            \def\operationtext{diminué}
            \def\calculation{\fpeval{\na * (1 - abs(\augmentation) / 100)}}
            
            \def\rounded{\fpeval{round(\na * (1 - abs(\augmentation) / 100 ), 2)}}
            \def\sign{-}
        }
    }
    \checkaugmentation
    Déterminer le nouveau prix d'un article coutant $\num{\na}~$\euro{} et dont le nouveau prix a une \operation\  de $\fpeval{abs(\augmentation)}~\%$
}

\end{multicols}
\vspace{-0.85cm}
\begin{multicols}{2}

\boiteQFQ{Question 3 :}{
    \def\he{3}
    \def\mi{11}
    \def\dia{61}
    
    
    %court
    Dans une série de $\he$ h $\mi$ min, il y a $\dia$ min de dialogues.\\
    Quel est le pourcentage de dialogues ?
}
\boiteQFA{
    \vspace{-0.35cm}
    \begin{itemize}[itemsep=0.2em]
        \item[\raisebox{-0.1cm}{\begin{itembox} \textbf{1.} \end{itembox}}] \def\n{8}
        \def\d{6}
        \def\nb{5}
        \def\db{9}
        $\Simplification{\fpeval{\n * \db + \nb * \d}}{\fpeval{\d * \db}}$
        \item[\raisebox{-0.1cm}{\begin{itembox} \textbf{2.} \end{itembox}}] \def\na{101.65}
        \def\augmentation{-49}
        \def\checkaugmentation{
            \ifthenelse{\augmentation > 0}
            { % Cas augmentation
                \def\operation{augmentation}
                \def\operationtext{augmenté}
                \def\calculation{\fpeval{\na * (1 + abs(\augmentation) / 100)}}
                \def\rounded{\fpeval{round(\na * (1 + abs(\augmentation) / 100), 2)}}
                
                \def\sign{+}
            }
            { % Cas réduction
                \def\operation{réduction}
                \def\operationtext{diminué}
                \def\calculation{\fpeval{\na * (1 - abs(\augmentation) / 100)}}
                
                \def\rounded{\fpeval{round(\na * (1 - abs(\augmentation) / 100 ), 2)}}
                \def\sign{-}
            }
        }
        \checkaugmentation
        \num{\rounded}
        \item[\raisebox{-0.1cm}{\begin{itembox} \textbf{3.} \end{itembox}}] \def\he{3}
        \def\mi{11}
        \def\dia{61}
        $\num{\fpeval{round(100 * \dia / (\he * 60 + \mi),1)}}~\%$
    \end{itemize}
}
\end{multicols}

\newpage

%\begin{multicols}{2}
\boiteQFdet{Solution détaillée de la question 1 :}{

\def\n{8}
\def\d{6}
\def\nb{5}
\def\db{9}


%court
Effectuer le calcul suivant en donnant le résultat sous forme simplifiée :
$$\dfrac{\n}{\d} + \dfrac{\nb}{\db}$$

\tikz{\draw[dashed, line width=1pt] (0,0) -- (\linewidth,0);}

\vspace{-0.25cm}\begin{multicols}{2}


\def\n{8}
\def\d{6}
\def\nb{5}
\def\db{9}


%court
Effectuer le calcul suivant en donnant le résultat sous forme simplifiée :
$$\dfrac{\n}{\d} + \dfrac{\nb}{\db}$$


%help

% Définitions des macros
\def\decnum{\Decomposition[Longue]{\fpeval{\n * \db + \nb * \d}}}
\def\decden{\Decomposition[Longue]{\fpeval{\d * \db}}}

\begin{align*}
    \dfrac{\n}{\d} + \dfrac{\nb}{\db}
    &=~ \dfrac{\n \times \db}{\d \times \db} + \dfrac{\nb \times \d}{\db \times \d}\\
    &=~ \dfrac{\fpeval{\n * \db + \nb * \d}}{\fpeval{\d * \db}}\\
    &=~ \dfrac{\decnum}{\decden}\\
    &=~ \Simplification{\fpeval{\n * \db + \nb * \d}}{\fpeval{\d * \db}}
\end{align*}

\end{multicols}
}

\newpage

\boiteQFdet{Solution détaillée de la question 2 :}{

\def\na{101.65}
\def\augmentation{-49}
\def\checkaugmentation{
\ifthenelse{\augmentation > 0}
{ % Cas augmentation
    \def\operation{augmentation}
    \def\operationtext{augmenté}
    \def\calculation{\fpeval{\na * (1 + abs(\augmentation) / 100)}}
    \def\rounded{\fpeval{round(\na * (1 + abs(\augmentation) / 100), 2)}}
    
    \def\sign{+}
}
{ % Cas réduction
    \def\operation{réduction}
    \def\operationtext{diminué}
    \def\calculation{\fpeval{\na * (1 - abs(\augmentation) / 100)}}
    
    \def\rounded{\fpeval{round(\na * (1 - abs(\augmentation) / 100 ), 2)}}
    \def\sign{-}
}
}
\checkaugmentation
Déterminer le nouveau prix d'un article coutant $\num{\na}~$\euro{} et dont le nouveau prix a une \operation\  de $\fpeval{abs(\augmentation)}~\%$

\tikz{\draw[dashed, line width=1pt] (0,0) -- (\linewidth,0);}


\vspace{-0.25cm}\begin{multicols}{2}

\def\na{101.65}
\def\augmentation{-49}
\def\checkaugmentation{
\ifthenelse{\augmentation > 0}
{ % Cas augmentation
    \def\operation{augmentation}
    \def\operationtext{augmenté}
    \def\calculation{\fpeval{\na * (1 + abs(\augmentation) / 100)}}
    \def\rounded{\fpeval{round(\na * (1 + abs(\augmentation) / 100), 2)}}
    
    \def\sign{+}
}
{ % Cas réduction
    \def\operation{réduction}
    \def\operationtext{diminué}
    \def\calculation{\fpeval{\na * (1 - abs(\augmentation) / 100)}}
    
    \def\rounded{\fpeval{round(\na * (1 - abs(\augmentation) / 100 ), 2)}}
    \def\sign{-}
}
}
\checkaugmentation
Déterminer le nouveau prix d'un article coutant $\num{\na}~$\euro{} et dont le nouveau prix a une \operation\  de $\fpeval{abs(\augmentation)}~\%$
\def\afficherresultat{
\ifthenelse{\equal{\calculation}{\rounded}}
{Nouveau Prix $= \num{\rounded}$\\ } % Si égaux
{Nouveau Prix $= \num{\calculation} \approx \num{\rounded}$\\
On donne l'arrondi au \textbf{centième} près car il s'agit d'un prix en euros.} % Si différents
}
\checkaugmentation
Pour \textbf{calculer} le \textbf{nouveau prix} après une \operation\  de $\fpeval{abs(\augmentation)}~\%$, on utilise la formule suivante : \\

Nouveau Prix $= \text{Ancien Prix} \sign \text{Ancien Prix} \times \left( \dfrac{t}{100}\right)  $ \\

où $t$ est le taux d'augmentation ou de réduction.

Ici il s'agit d'une \operation, avec $\text{Ancien Prix} = \num{\na}$ et $t = \num{\fpeval{abs(\augmentation)}}$.\\

Nouveau Prix $= \num{\na} \sign \num{\na} \times \dfrac{\num{\fpeval{abs(\augmentation)}}}{100}$\\

Nouveau Prix $=\num{\na} \sign \num{\fpeval{\na * abs(\augmentation) / 100}} $\\

Ainsi  :\\
\afficherresultat

\end{multicols}
}

%\columnbreak
\newpage


\boiteQFdet{Solution détaillée de la question 3 :}{

\def\he{3}
\def\mi{11}
\def\dia{61}


%court
Dans une série de $\he$ h $\mi$ min, il y a $\dia$ min de dialogues.\\
Quel est le pourcentage de dialogues ?

\tikz{\draw[dashed, line width=1pt] (0,0) -- (\linewidth,0);}

\vspace{-0.25cm}\begin{multicols}{2}

\def\he{3}
\def\mi{11}
\def\dia{61}


%court
Dans une série de $\he$ h $\mi$ min, il y a $\dia$ min de dialogues.
Quel est le pourcentage de dialogues ?\\



%help

La série dure $\num{\fpeval{\he * 60 + \mi}}~$minutes.\\

\begin{tcbtab}[Tableau de proportionnalité]{p{2.5cm}|p{2cm}|p{2cm}}
\cellcolor{gray!10}\bfseries Quantité & $\dia$ & \color{red} \bfseries \num{\fpeval{round(100 * \dia / (\he * 60 + \mi),1)}}\\\hline
\cellcolor{gray!10}\bfseries Total& \num{\fpeval{\he * 60 + \mi}} & 100\\
\end{tcbtab}

Par produit en croix : \\

$100 \times \dia \div \num{\fpeval{\he * 60 + \mi}} = \num{\fpeval{round(100 * \dia / (\he * 60 + \mi),1)}}$



Ainsi, $\num{\fpeval{\he * 60 + \mi}}~$minutes de dialogues représentent $\num{\fpeval{round(100 * \dia / (\he * 60 + \mi),1)}}~\%$ de la série.

\end{multicols}
}
%\end{multicols}