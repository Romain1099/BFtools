%niveau estimé : 4ème
    %date de projection : 11_10_2024
    %themes abordés : Évaluation, Pythagore, Fractions.
    
    \vspace{-0.5cm}
    \begin{multicols}{2}
        \boiteQFQ{Question 1 :}{
            \textbf{Calculer} la valeur de l'expression
            \[
                3x^2 + 5x - 7
            \]
            en remplaçant $x$ par $4$.
        }
        \boiteQFQ{Question 2 :}{
            \textbf{Écrire} l'\textbf{égalité} de \textbf{Pythagore} pour le triangle $KLM$, rectangle en $M$.
        }
    \end{multicols}
    \vspace{-0.85cm}
    \begin{multicols}{2}
        
        \boiteQFQ{Question 3 :}{
            \textbf{Calculez} la somme des \textbf{fractions} : \\$\dfrac{5}{3}$ et $\dfrac{2}{7}$\\ \textbf{Exprimez} le résultat sous forme d'une \textbf{fraction} simplifiée.
        }
        \boiteQFA{}
    \end{multicols}
    
    \newpage
    \vspace{-0.5cm}
    \begin{multicols}{2}
        \boiteQFQ{Question 1 :}{
            \textbf{Calculer} la valeur de l'expression
            \[
                3x^2 + 5x - 7
            \]
            en remplaçant $x$ par $4$.
        }
        \boiteQFQ{Question 2 :}{
            \textbf{Écrire} l'\textbf{égalité} de \textbf{Pythagore} pour le triangle $KLM$, rectangle en $M$.
        }
        
    \end{multicols}
    \vspace{-0.85cm}
    \begin{multicols}{2}
        
        \boiteQFQ{Question 3 :}{
            \textbf{Calculez} la somme des \textbf{fractions} : \\$\dfrac{5}{3}$ et $\dfrac{2}{7}$\\ \textbf{Exprimez} le résultat sous forme d'une \textbf{fraction} simplifiée.
        }
        \boiteQFA{
            \vspace{-0.35cm}
            \begin{itemize}[itemsep=0.2em]
                \item[\raisebox{-0.2cm}{\begin{itembox} \textbf{Q.1} \end{itembox}}] $61$
            \end{itemize}
        }
    \end{multicols}
    \newpage
    \vspace{-0.5cm}
    \begin{multicols}{2}
        \boiteQFQ{Question 1 :}{
            \textbf{Calculer} la valeur de l'expression
            \[
                3x^2 + 5x - 7
            \]
            en remplaçant $x$ par $4$.
        }
        \boiteQFQ{Question 2 :}{
            \textbf{Écrire} l'\textbf{égalité} de \textbf{Pythagore} pour le triangle $KLM$, rectangle en $M$.
        }
        
    \end{multicols}
    \vspace{-0.85cm}
    \begin{multicols}{2}
        
        \boiteQFQ{Question 3 :}{
            \textbf{Calculez} la somme des \textbf{fractions} : \\$\dfrac{5}{3}$ et $\dfrac{2}{7}$\\ \textbf{Exprimez} le résultat sous forme d'une \textbf{fraction} simplifiée.
        }
        \boiteQFA{
            \vspace{-0.35cm}
            \begin{itemize}[itemsep=0.2em]
                \item[\raisebox{-0.2cm}{\begin{itembox} \textbf{Q.1} \end{itembox}}] $61$
                \item[\raisebox{-0.2cm}{\begin{itembox} \textbf{Q.2} \end{itembox}}] $KL^2 = KM^2 + LM^2$
            \end{itemize}
        }
    \end{multicols}
    \newpage
    
    \begin{multicols}{2}
        \boiteQFQ{Question 1 :}{
            \textbf{Calculer} la valeur de l'expression
            \[
                3x^2 + 5x - 7
            \]
            en remplaçant $x$ par $4$.
        }
        \boiteQFQ{Question 2 :}{
            \textbf{Écrire} l'\textbf{égalité} de \textbf{Pythagore} pour le triangle $KLM$, rectangle en $M$.
        }
        
    \end{multicols}
    \vspace{-0.85cm}
    \begin{multicols}{2}
        
        \boiteQFQ{Question 3 :}{
            \textbf{Calculez} la somme des \textbf{fractions} : \\$\dfrac{5}{3}$ et $\dfrac{2}{7}$\\ \textbf{Exprimez} le résultat sous forme d'une \textbf{fraction} simplifiée.
        }
        \boiteQFA{
            \vspace{-0.35cm}
            \begin{itemize}[itemsep=0.2em]
                \item[\raisebox{-0.2cm}{\begin{itembox} \textbf{Q.1} \end{itembox}}] $61$
                \item[\raisebox{-0.2cm}{\begin{itembox} \textbf{Q.2} \end{itembox}}] $KL^2 = KM^2 + LM^2$
                \item[\raisebox{-0.2cm}{\begin{itembox} \textbf{Q.3} \end{itembox}}] $\dfrac{41}{21}$
            \end{itemize}
        }
    \end{multicols}
    
    \newpage
    
    %\begin{multicols}{2}
    \boiteQFdet{Question 1 :}{
        
        \textbf{Calculer} la valeur de l'expression
        \[
            3x^2 + 5x - 7
        \]
        en remplaçant $x$ par $4$.
        
        \tikz{\draw[dashed, line width=1pt] (0,0) -- (\linewidth,0);}
        
        \vspace{-0.25cm}\begin{multicols}{2}
        
        
        \textbf{Remplacer} $x$ par $4$ dans l'expression $3x^2 + 5x - 7$ :
        
        \[
            3\times(4)^2 + 5\times(4) - 7
        \]
        
        \textbf{Calculer} en effectuant les opérations :
        
        \[
            = 3 \times 16 + 20 - 7
        \]
        
        \[
            = 48 + 20 - 7
        \]
        
        \[
            = 68 - 7
        \]
        
        \[
            = 61
        \]
        
        La \textbf{valeur} de l'expression est donc $\mathbf{61}$.
        
    \end{multicols}
}

\newpage

\boiteQFdet{Question 2 :}{
    
    \textbf{Écrire} l'\textbf{égalité} de \textbf{Pythagore} pour le triangle $KLM$, rectangle en $M$.
    
    \tikz{\draw[dashed, line width=1pt] (0,0) -- (\linewidth,0);}
    
    
    \vspace{-0.25cm}\begin{multicols}{2}
    
    Pour le \textbf{triangle rectangle} $KLM$ avec l'angle droit en $M$, le segment $[KL]$ est l'\textbf{hypoténuse}. \\L'\textbf{égalité de Pythagore} est donc :
    \[
        KL^2 = KM^2 + LM^2
    \]
    
\end{multicols}
}

%\columnbreak
\newpage

\boiteQFdet{Question 3 :}{

\textbf{Calculez} la somme des \textbf{fractions} : \\$\dfrac{5}{3}$ et $\dfrac{2}{7}$\\ \textbf{Exprimez} le résultat sous forme d'une \textbf{fraction} simplifiée.

\tikz{\draw[dashed, line width=1pt] (0,0) -- (\linewidth,0);}

\vspace{-0.25cm}\begin{multicols}{2}

La \textbf{somme} de deux \textbf{fractions} se calcule en trouvant un dénominateur commun.\\ Pour $\dfrac{5}{3}$ et $\dfrac{2}{7}$, le \textbf{plus petit commun multiple} des dénominateurs $3$ et $7$ est $21$. Nous devons donc \textbf{réduire les fractions au même dénominateur} :\\
$
\dfrac{5}{3} = \dfrac{5 \times 7}{3 \times 7} = \dfrac{35}{21}$
$
\dfrac{2}{7} = \dfrac{2 \times 3}{7 \times 3} = \dfrac{6}{21}$\\

\textbf{Additionnons} ensuite ces deux \textbf{fractions} :\\
$
\dfrac{35}{21} + \dfrac{6}{21} = \dfrac{35 + 6}{21} = \dfrac{41}{21}$

La fraction $\dfrac{41}{21}$ ne peut pas être \textbf{simplifiée} davantage, car $41$ et $21$ n'ont pas de diviseur commun autre que $1$. Ainsi, la somme souhaitée est $\dfrac{41}{21}$.

\end{multicols}
}
%\end{multicols}