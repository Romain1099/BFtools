%niveau estimé : 4ème
    %date de projection : 15_10_2024
    %themes abordés : Divisors., Calcul, Calcul.
    
    \vspace{-0.5cm}
    \begin{multicols}{2}
        \boiteQFQ{Question 1 :}{
            \textbf{Déterminer} la liste des \textbf{diviseurs} de $30$.
        }
        \boiteQFQ{Question 2 :}{
            \textbf{Compléter} l'opération suivante en respectant les priorités : $(-8) \times (3 + 5) \div (-4)$.
        }
    \end{multicols}
    \vspace{-0.85cm}
    \begin{multicols}{2}
        
        \boiteQFQ{Question 3 :}{
            Calculez le \textbf{nouveau prix} d'un article coûtant $12\,\text{\euro}$ après une \textbf{augmentation} de $20\%$.
        }
        \boiteQFA{}
    \end{multicols}
    
    \newpage
    \vspace{-0.5cm}
    \begin{multicols}{2}
        \boiteQFQ{Question 1 :}{
            \textbf{Déterminer} la liste des \textbf{diviseurs} de $30$.
        }
        \boiteQFQ{Question 2 :}{
            \textbf{Compléter} l'opération suivante en respectant les priorités : $(-8) \times (3 + 5) \div (-4)$.
        }
        
    \end{multicols}
    \vspace{-0.85cm}
    \begin{multicols}{2}
        
        \boiteQFQ{Question 3 :}{
            Calculez le \textbf{nouveau prix} d'un article coûtant $12\,\text{\euro}$ après une \textbf{augmentation} de $20\%$.
        }
        \boiteQFA{
            \vspace{-0.35cm}
            \begin{itemize}[itemsep=0.2em]
                \item[\raisebox{-0.2cm}{\begin{itembox} \textbf{Q.1} \end{itembox}}] $1,\, 2,\, 3,\, 5,\, 6,\, 10,\, 15,\, 30$
            \end{itemize}
        }
    \end{multicols}
    \newpage
    \vspace{-0.5cm}
    \begin{multicols}{2}
        \boiteQFQ{Question 1 :}{
            \textbf{Déterminer} la liste des \textbf{diviseurs} de $30$.
        }
        \boiteQFQ{Question 2 :}{
            \textbf{Compléter} l'opération suivante en respectant les priorités : $(-8) \times (3 + 5) \div (-4)$.
        }
        
    \end{multicols}
    \vspace{-0.85cm}
    \begin{multicols}{2}
        
        \boiteQFQ{Question 3 :}{
            Calculez le \textbf{nouveau prix} d'un article coûtant $12\,\text{\euro}$ après une \textbf{augmentation} de $20\%$.
        }
        \boiteQFA{
            \vspace{-0.35cm}
            \begin{itemize}[itemsep=0.2em]
                \item[\raisebox{-0.2cm}{\begin{itembox} \textbf{Q.1} \end{itembox}}] $1,\, 2,\, 3,\, 5,\, 6,\, 10,\, 15,\, 30$
                \item[\raisebox{-0.2cm}{\begin{itembox} \textbf{Q.2} \end{itembox}}] $16 $
            \end{itemize}
        }
    \end{multicols}
    \newpage
    
    \begin{multicols}{2}
        \boiteQFQ{Question 1 :}{
            \textbf{Déterminer} la liste des \textbf{diviseurs} de $30$.
        }
        \boiteQFQ{Question 2 :}{
            \textbf{Compléter} l'opération suivante en respectant les priorités : $(-8) \times (3 + 5) \div (-4)$.
        }
        
    \end{multicols}
    \vspace{-0.85cm}
    \begin{multicols}{2}
        
        \boiteQFQ{Question 3 :}{
            Calculez le \textbf{nouveau prix} d'un article coûtant $12\,\text{\euro}$ après une \textbf{augmentation} de $20\%$.
        }
        \boiteQFA{
            \vspace{-0.35cm}
            \begin{itemize}[itemsep=0.2em]
                \item[\raisebox{-0.2cm}{\begin{itembox} \textbf{Q.1} \end{itembox}}] $1,\, 2,\, 3,\, 5,\, 6,\, 10,\, 15,\, 30$
                \item[\raisebox{-0.2cm}{\begin{itembox} \textbf{Q.2} \end{itembox}}] $16 $
                \item[\raisebox{-0.2cm}{\begin{itembox} \textbf{Q.3} \end{itembox}}] $\text{Nouveau prix} = 12 \times 1{,}20 = 14{,}40 \,\text{\euro} $
            \end{itemize}
        }
    \end{multicols}
    
    \newpage
    
    %\begin{multicols}{2}
    \boiteQFdet{Question 1 :}{
        
        \textbf{Déterminer} la liste des \textbf{diviseurs} de $30$.
        
        \tikz{\draw[dashed, line width=1pt] (0,0) -- (\linewidth,0);}
        
        \vspace{-0.25cm}\begin{multicols}{2}
        
        
        Pour \textbf{déterminer} les \textbf{diviseurs} d'un nombre entier, il suffit de \textbf{trouver} les produits qui donnent ce nombre. Prenons le nombre $30$ par exemple.
        
        On cherche les \textbf{produits} qui donnent $30$ :
        
        \begin{itemize}
            \item $1 \times 30 = 30$,
            \item $2 \times 15 = 30$,
            \item $3 \times 10 = 30$,
            \item $5 \times 6 = 30$.
        \end{itemize}
        
        Ainsi, les \textbf{diviseurs} de $30$ sont les nombres $1$, $2$, $3$, $5$, $6$, $10$, $15$, et $30$.
        
    \end{multicols}
}

\newpage

\boiteQFdet{Question 2 :}{
    
    \textbf{Compléter} l'opération suivante en respectant les priorités : $(-8) \times (3 + 5) \div (-4)$.
    
    \tikz{\draw[dashed, line width=1pt] (0,0) -- (\linewidth,0);}
    
    
    \vspace{-0.25cm}\begin{multicols}{2}
    
    L'opération donnée est $(-8) \times (3 + 5) \div (-4)$. Pour \textbf{compléter} cette opération, respectons les \textbf{priorités} des opérations :
    
    1. \textbf{Calculer} d'abord l'expression entre parenthèses : $3 + 5 = 8$.
    
    \[
        (-8) \times 8 \div (-4)
    \]
    
    2. \textbf{Effectuer} ensuite la multiplication : \\
    $(-8) \times 8 = -64$
    
    \[
        -64 \div (-4)
    \]
    
    3. \textbf{Calculer} enfin la division :
    
    \[
        -64 \div (-4) = 16
    \]
    
    Ainsi, le résultat de l'opération est $16$.
    
\end{multicols}
}

%\columnbreak
\newpage

\boiteQFdet{Question 3 :}{

Calculez le \textbf{nouveau prix} d'un article coûtant $12\,\text{\euro}$ après une \textbf{augmentation} de $20\%$.

\tikz{\draw[dashed, line width=1pt] (0,0) -- (\linewidth,0);}

\vspace{-0.25cm}\begin{multicols}{2}

Le \textbf{nouveau prix} après une \textbf{augmentation} de \(20\%\) revient à \textbf{calculer} le \textbf{quotient} de l'\textbf{augmentation} par le \textbf{prix initial}, et à l'\textbf{ajouter} au \textbf{prix initial}.

\begin{enumerate}
    \item \textbf{Calculer} l'\textbf{augmentation} :
    
    \[
    \text{Augmentation} = 12 \times \dfrac{20}{100} \]
    \[ \text{Augmentation} = 12 \times 0{,}20 = \num{\fpeval{12*0.2}}
    \]
    
    \item \textbf{Nouveau prix} :
    
    \[
        \text{Nouveau prix} = 12 + \num{\fpeval{12*0.2}} = \num{\fpeval{12+12*0.2}}\,\text{\euro}
    \]
\end{enumerate}

Ainsi, le \textbf{nouveau prix} de l'article est \(\mathbf{\num{\fpeval{12+12*0.2}}}\,\text{\euro}\).

\end{multicols}
}
%\end{multicols}