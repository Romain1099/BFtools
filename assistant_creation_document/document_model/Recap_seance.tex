\documentclass[a4paper,11pt,fleqn]{article}

\usepackage[left=1cm,right=1cm,top=0.5cm,bottom=2cm]{geometry}

\usepackage{bfcours}
\usepackage{bfcours-fonts}
%\usepackage{bfcours-fonts-dys}

\def\rdifficulty{1}
\setrdexo{%left skip=1cm,
display exotitle,
exo header = tcolorbox,
%display tags,
skin = bouyachakka,
lower ={box=crep},
display score,
display level,
save lower,
score=\points,
level=\rdifficulty,
overlay={\node[inner sep=0pt,
anchor=west,rotate=90, yshift=0.3cm]%,xshift=-3em], yshift=0.45cm
at (frame.south west) {\thetags[0]} ;}
]%obligatoire
}
\setrdcrep{seyes, correction=true, correction color=monrose, correction font = \large\bfseries}


\hypersetup{
    pdfauthor={R.Deschamps},
    pdfsubject={},
    pdfkeywords={},
    pdfproducer={LuaLaTeX},
    pdfcreator={Boum Factory}
}
\newcommand{\messageIntro}[1]{
    Rapport d'incident, \hfill #1\\
}
\renewcommand{\punition}[4]{
    Travail supplémentaire, à rendre pour #2.\\
    Ce travail consiste à :\\
    \begin{center}#4\end{center}
}
\newcommand{\retenue}[4]{
    Retenue d'une durée de #1.
    La retenue aura lieu #2 en salle #3 durant laquelle le travail suivant sera proposé :\\
    \begin{center}#4\end{center}
}
\begin{document}

\setcounter{pagecounter}{0}
\setcounter{ExoMA}{0}

%Pour les overlay
\def\points{\phantom{AAA}}
\def\difficulty{\phantom{AAA}}
\chapitre[
    % niveau : $\mathbf{6^{\text{ème}}}$,$\mathbf{5^{\text{ème}}}$,$\mathbf{4^{\text{ème}}}$,$\mathbf{3^{\text{ème}}}$,$\mathbf{2^{\text{nde}}}$,$\mathbf{1^{\text{ère}}}$,$\mathbf{T^{\text{Le}}}$
    ]{
    %
    }{
    % type_etablissement : Collège,Lycée
    }{
    % nom_etablissement : Amadis Jamyn,Eugène Belgrand
    }{
    %
    }{
    % type_document : commande(intitule_seance())
    }

\begin{tcolorbox}[blank]
	\begin{Colonnes}{3}
		\begin{bfbox}{Lieu et date :}
			\textbf{Lieu :} % Lieu : Salle 203,Salle A1,Cours de récréation,couloirs,réfectoire,dans les rangs,pendant la montée,
		    
		    
			\textbf{Créneau :} % creneau : M1,M2,M3,M4,S1,S2,S3,S4
		\end{bfbox}
		\begin{bfbox}{Informations sur la séance :}[raster multicolumn=2]
			\textbf{Titre :} % Titre_seance : ,
		
		
		
			\textbf{Thème :} % Theme_seance : ,
		
		
		
			\textbf{Type de séance :} % type_seance : Cours, Exercices, Evaluation, Activité
		\end{bfbox}
	\end{Colonnes}
\end{tcolorbox}
\begin{bfbox}{Description de la séance :}

	\textbf{Automatismes :} % Automatismes : ,
	
	\textbf{Travail à faire pour la séance :} 
	\begin{itemize}[label=$\bullet$]
	    % travail_a_faire : ,
	\end{itemize}
	
	\textbf{Contenu :} 
	\begin{itemize}[label=$\bullet$]
	    % contenu_seance_principal : ,

	\end{itemize}
	\textbf{Travail à faire à l'issue de la séance :} 
	\begin{itemize}[label=$\bullet$]
	    % devoirs_prochaine : ,
	\end{itemize}
	\vspace{2cm}
\end{bfbox}

\begin{bfbox}{Remarques sur la séance :}

\textbf{Remarques générales :} 






\textbf{Remarques sur le matériel :}





\textbf{Remarques sur la discipline :}




\end{bfbox}

\end{document}